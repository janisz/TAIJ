\section{Maszyny Turinga}
\begin{df}[Maszyna Turinga - model podstawowy]~\\
	Maszyną Turinga w modelu podstawowym nazywamy system 
	\begin{equation}
		M = (Q, \Sigma, \Gamma, \delta, q_0, B, F)
	\end{equation}
	\begin{description}
		\item[$Q$] -- skończony zbiór stanów
		\item[$\Sigma$] -- zbiór symboli wejściowych
		\item[$\Gamma$] -- skończony zbiór wszystkich symboli będący alfabetem taśmy
		\item[$\delta$] -- funkcja przejścia $\delta = Q\times\Gamma \rightarrow Q\times\Gamma\times\{L,P\}$
		\item[$q_0$] -- stan początkowy $q_0 \in Q$
		\item[$B$] -- zdefiniowany symbol pusty $B\in\Gamma$
		\item[$F$] -- zbiór stanów akceptujących $F\subset Q$
	\end{description}
	\begin{uwaga}
		Obliczenie Maszyny Turinga polega na wykonaniu kolejnych ruchów określonych funkcją przejścia
	\end{uwaga}		
	\begin{uwaga}
		Obliczenie może się zakończyć ale może też być kontynuowane w nieskończoność
	\end{uwaga}
\end{df}

\begin{df}[Relacja ruchu]
	Obliczeniem Maszyny Turinga nazywamy ciąg opisów chwilowych pozostających w relacji
\end{df}

\begin{df}[Maszyna Turinga z własnością stopu]~\\
	Maszyny Turinga które zawsze kończą obliczenie niezależnie od danych nazywamy Maszynami Turinga 
	z własnością stopu albo algorytmami
\end{df}

\begin{df}[Maszyna Turinga z własnością stopu ze stanem akceptującym]~\\
	Maszyną Turinga z własnością stopu ze stanem akceptującym nazywamy system
	\begin{equation}
		M = (Q, \Sigma, \Gamma, \delta, q_0, B, F = \{ACC\})
	\end{equation}
	Przy czym istnieje założenie, że przejście do stanu akceptującego jest zakończeniem obliczeń
\end{df}

\begin{df}[Maszyna Turinga z własnością stopu]~\\
	Maszyną Turinga z własnością stopu nazywamy system 
	\begin{equation}
		M = (Q, \Sigma, \Gamma, \delta, q_0, B, F = \{ACC\}, N = \{REJ\})
	\end{equation}
	gdzie $N$ to zbiór stanów odrzucających.
	\begin{itemize}
		\item Zakończenie obliczeń jest równoważne przejściu do stanu akceptującego lub odrzucającego
		\item Maszyna kończy każde swoje obliczenie
	\end{itemize}
	\begin{uwaga}
		Maszyna nie akceptuje wszystkich języków akceptowanych przez Maszynę Turinga w modelu 
		podstawowym, jet jej szczególnym przypadkiem
	\end{uwaga}
\end{df}

\begin{df}[Maszyna Turinga z wartownikiem]~\\
	Maszyną Turinga z wartownikiem nazywamy system $M = (Q, \Sigma, \Gamma, \delta, q_0, B, F)$
	z tym że alfabet taśmy posiada specjalny symbol $\#$ nazywany wartownikiem, zapisany w pierwszej 
	komórce taśmy.
	\begin{uwaga}
		Ta maszyna jest równoważna Maszynie Turinga w modelu podstawowym
	\end{uwaga}
\end{df}

\begin{df}[Maszyna Turinga z taśmą wielościeżkową]~\\
	Maszyną Turinga z taśmą wielościeżkową nazywamy system $M = (Q, \Sigma, \Gamma, \delta, q_0, B, F)$,
	gdzie funkcja przejścia zdefiniowana jest następująco
	\begin{equation}
		\delta : Q\times\Gamma^k\rightarrow G\times\Gamma\times\{L,P\}
	\end{equation}
\end{df}

\begin{tw}
	Maszyna Turinga z taśmą wielościeżkową jest równoważna Maszynie Turinga w modelu podstawowym (jednościeżkowej)
\begin{uwaga}
	Można ją zastąpić maszyną jednościeżkową kosztem zwiększenia alfabetu taśmy, który będzie
	iloczynem kartezjańskim alfabetów ścieżek.
	Aby formalnie udowodnić równoważność tych maszyn należałoby skonstruować maszynę symulującą
	działanie drugiej podobnie jak w Maszynie Turinga z wartownikiem.
\end{uwaga}
\end{tw}

\begin{tw}
	Maszyny Turinga z taśmą obustronnie nieograniczoną są równoważne Maszyną Turinga w modelu podstawowym
	
	\begin{proof}
	Wystarczy pokazać że dla maszyny jednego typu jesteśmy w stanie stworzyć maszynę drugiego typu akceptującą ten sam
	język.
	\begin{itemize}
		\item Maszyna z taśmą obustronnie nieograniczoną symulująca pracę maszyny w modelu podstawowym byłaby
		zdefiniowana identycznie jak symulowana maszyna.	W oczywisty sposób akceptowałaby ten sam język.
		\item Symulacja maszyny z taśmą obustronnie nieograniczoną przez maszynę w modelu podstawowym. 
		Ponieważ Maszyna Turinga wielościeżkowa jest równoważna maszynie w modelu podstawowym, pokażemy
		równoważność do wielościeżkowej. Zatem maszyna symulująca będzie maszyną dwuścieżkową, dodatkowo można pokusić się o 
		wprowadzenie wartownika. Koncept na realizację maszyny symulującej jest następujący:
			\begin{enumerate}
				\item Jeśli maszyna symulowana prowadzi obliczenia na lewo od pierwszego znaku lub wartownika
				to obliczenia prowadzone są na dolnej taśmie maszyny symulującej.
				\item W przeciwnym przypadku obliczenia prowadzone są na górnej taśmie.
				\item Należy wziąć pod uwagę inwersję ruchów w lewo/prawo na dolnej taśmie
			\end{enumerate}
	\end{itemize}
	\end{proof}
\end{tw}

\begin{df}[Wielotaśmowa Maszyna Turinga]~\\
	k-taśmową Maszyną Turinga nazywamy system
	\begin{equation}
		M = (Q, \Sigma, \Gamma_1\times\dots\times\Gamma_k, \delta, q_0, B, F)
	\end{equation}
	gdzie funkcja przejścia zdefiniowana jest następująco
	\begin{equation}
		\delta: Q\times(\Gamma_1\times\dots\times\Gamma_k) \rightarrow Q\times(\Gamma_1\times\dots\Gamma_k)\times\{L,R,S\}^k
	\end{equation}
\end{df}

\begin{tw}
	Klasa Maszyn Turinga wielotaśmowych jest równoważna klasie Maszyn Turinga w modelu podstawowym
	\begin{proof}
		Każda maszyna jednotaśmowa jest szczególnym przypadkiem maszyny wielotaśmowej. Ponieważ maszyna obustronnie
		ograniczona jest równoważna podstawowej to implikacja w jedną stronę jest trywialna.\\\\
		Pokażemy symylację wielotaśmowej maszyny yżywając Maszyny Turinga z taśmą obustronnie nieograniczoną o $2k$ 
		ścieżkach. Każdej taśmie będzie odpowiadać para ścieżek maszyny symulującej.
		Górna ścieżka każdej pary będzie zawierać symbol oznaczający położenie głowicy w maszynie symulowanej natomiast
		dolna będzie zawierać symulowaną taśmę.
		Symulacja będzie przebiegać następująco
		\begin{enumerate}
			\item Głowica maszyny symulującej znajduje się nad najbardziej na lewo wysuniętym symbolem -- znacznikiem głowic
			\item Głowica maszyny symulującej przesuwa się w prawo aż do najbardziej na prawo wysuniętego znacznika głowic.
			Zapamiętuje przy tym informację o symbolach czytanych przez głowicę maszyny wielotaśmowej -- 
			potrzebny jest odpowiednio duży zbiór stanów.
			\item Mając informację o stanie i symbolach pod głowicami maszyny wielotaśmowej, maszyna 
			symulująca zapamiętuje pełną informację o stanie Maszyny wielotaśmowej w odpowiadającym mu stanie.
			Oznacza to że maszyna symulująca jest teraz w stanie wyznaczyć opis chwilowy maszyny symulowanej 
			przed wykonaniem ruchu. Następnie stosując funkcje przejścia może wyznaczyć opis chwilowy po wykonaniu ruchu.
			\item Głowica maszyny symulującej przesuwa się do najbardziej na lewo wysuniętego symbolu oznaczającego położenie
			głowicy w modelu wielotaśmowym (znacznik głowicy). W trakcie tego powrotu maszyna symulująca uaktualnia odpowiednie
			komórki ścieżek opisujące taśmy maszyny symulowanej, a także położenie znaczników głowic.
			\item Głowica maszyny symulującej zatrzyma się na najbardziej na lewo położonym znacznikiem głowic 
			oraz zmieni stan zgodnie ze mianą stanu maszyny symulowanej.
		\end{enumerate}
	\end{proof}
	\begin{uwaga}
		Symulacja maszyny wielotaśmowej jednotaśmową skutkuje wzrostem złożoności obliczeniowej z kwadratem ruchów.
	\end{uwaga}
\end{tw}

\begin{df}[Niedeterministyczna Maszyna Turinga]~\\
	Definicja jest analogiczna do Maszyny Turinga w modelu podstawowym z tym że funkcja przejścia
	zdefiniowana jest następująco
	\begin{equation}
		\delta: Q\times\Gamma \rightarrow \bigcup^\infty_{k=0} (Q\times\Gamma\times\{L,R\})^k\quad
		\footnote{Dla $k=0$ wartość jest nieokreślona}
	\end{equation}
	Zatem ruch tej maszyny polega na niedeterministycznym wyborze jednej z wartości funkcji przejścia,
	a następnie wykonaniu ruchu takiego jaki wykonałaby deterministyczna maszyna.
	\begin{uwaga}
		Obliczenie niedeterministycznej Maszyny Turinga jest drzewem w którym
		\begin{itemize}
			\item wierzchołki etykietowane są opisem chwilowym maszyny	
			\item potomkami dowolnego wierzchołka są wierzchołki pozostające z nim w relacji ruchu
		\end{itemize}
	\end{uwaga}
\end{df}
