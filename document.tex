\documentclass[12pt,a4paper]{article}

\usepackage{fullpage}

\usepackage[utf8]{inputenc}
\usepackage[T1]{fontenc}
\usepackage[a4paper]{geometry}

\usepackage{pdflscape}
\usepackage{float}

\usepackage{color}

\usepackage{amsthm}
\usepackage{amsfonts}
\usepackage{amsmath} %pakiet potrzebny do \eqref
\usepackage{amssymb} %pakiet potrzebny do \blacksquare i \mathbb{•}

\usepackage[polish]{babel}
\usepackage{polski}

\usepackage{graphicx}

\usepackage{listliketab}

\newtheorem{lemat}{Lemat}
\newtheorem{tw}{Twierdzenie}
\newtheorem{przyklad}{Przykład}
\theoremstyle{definition}
\newtheorem{df}{Definicja}

\usepackage[pdfauthor={Janisz},%
pdftitle={Teoria Automatów i Języków},%
pagebackref=true,%
pdftex]{hyperref}
\hypersetup{colorlinks=false}


\pagestyle{plain}
\begin{document}
\title{ Teoria Automatów i Języków}
\author{\vspace{-5ex}}
\date{\today}
\maketitle
\tableofcontents

\DeclareGraphicsExtensions{.pdf,.png,.jpg}
\begin{center}
\leavevmode

\vfill

\includegraphics[width=1 in]{by-sa.png}
\end{center}
\label{fig:cc}
%insert a link to the licence and its description below
\scriptsize{Ten utwór jest dostępny na licencji  
\href{http://creativecommons.org/licenses/by-sa/3.0/pl/}{Creative Commons Uznanie autorstwa-Na tych samych warunkach 3.0 Polska.}}

\pagebreak

\section{Relacje}

\begin{df}
Niech $X_1, X_2, \dots, X_n$ będą zbiorami. Podzbiór iloczynu kartezjańskiego $R \subset X_1 \times X_2 \times \dots \times X_n$ 
nazywamy n-argumentową relacją. 
\end{df}

\begin{df}
Niech $X, Y$ będą zbiorami. Wtedy relacją dwuargumentową nazywamy $\rho = X \times Y$, a zbiory $X, Y$ odpowiednio dziedziną i przeciwdziedziną.  
\end{df}

\begin{df}
Relacją binarną nazywamy taką relację dwuargumentową, w której dziedzina i przeciwdziedzina są równe. $\rho = X \times X$
\end{df}

\subsection{Własności relacji}

$\forall x,y,z \in X$
\begin{listliketab}
\storestyleof{itemize} 
\begin{tabular}{ll}
	\textbullet  zwrotna & $x \rho x$ \\
	\textbullet  przeciw-zwrotna & $\neg x\rho x$ \\
	\textbullet  symetryczna & $x\rho y \Rightarrow y\rho x$ \\
	\textbullet  przeciw-symetryczna & $x \rho y \Rightarrow \neg y\rho x $ \\
	\textbullet  antysymetryczna & $x\rho y \wedge y\rho x \Rightarrow x\rho z$ \\
	\textbullet  przechodnia & $x\rho y \wedge y\rho z \Rightarrow x\rho z$ \\
	\textbullet  spójna & $ x\rho y \vee y\rho x $ \\
\end{tabular}
\end{listliketab}

\begin{df}
Relację nazywamy relacją równoważności jeśli jest jednocześnie zwrotna, symetryczna i przechodnia
\end{df}

\begin{df}
Niech $\rho\subset X\times X$ będzie relacją binarną, a $\mathcal{R}$ zbiorem własności relacji. Powiemy że $\rho'$ jest domknięciem relacji $\rho$
ze względu na $\mathcal{R} \quad \Leftrightarrow$
\begin{enumerate}
	\item $\rho \subset \rho'$
	\item $\rho'$ jest domknięta ze względu na własności z $\mathcal{R}$
	\item $\rho'$ jest najmniejszą relacją spełniającą powyższe warunki
\end{enumerate}
\end{df}

\begin{przyklad}
	$\rho \subset \mathbb{N}\times\mathbb{N} \quad m,n \in \mathbb{N} \quad m\rho n \equiv m +1 = n$ \\
	\begin{center}
	\begin{tabular}{ccccccccc}
		$\rho$ & 0 & 1 & 2 & 3 & 4 & 5 & 6 & \dots \\
			0  & {\color{red}1} & {\color{blue}1} & {\color{green}1}  & {\color{green}1}  & {\color{green}1}  & {\color{green}1}  & {\color{green}1}  & \dots \\
			1  & {\color{yellow}0}  & {\color{red}1}  & {\color{blue}1} & {\color{green}1}  & {\color{green}1}  & {\color{green}1}  & {\color{green}1}  & \dots \\
			2  & {\color{yellow}0}  & {\color{yellow}0}  & {\color{red}1}  & {\color{blue}1} & {\color{green}1}  &  {\color{green}1} & {\color{green}1}  & \dots \\
			3  & {\color{yellow}0}  & {\color{yellow}0}  & {\color{yellow}0}  & {\color{red}1}  & {\color{blue}1} & {\color{green}1}  & {\color{green}1}  & \dots \\
			4  & {\color{yellow}0}  & {\color{yellow}0}  & {\color{yellow}0}  & {\color{yellow}0}  & {\color{red}1}  & {\color{blue}1} & {\color{green}1}  & \dots \\
			5  & {\color{yellow}0}  & {\color{yellow}0}  & {\color{yellow}0}  & {\color{yellow}0}  & {\color{yellow}0}  & {\color{red}1}  & {\color{blue}1} & \dots \\
	     \dots & \dots  & \dots  & \dots  & \dots  & \dots  & \dots  & {\color{red}1}  & \dots \\
	\end{tabular}\\
	\end{center}	
	{\color{yellow} $0$ oznacza brak relacji}, $1$ oznacza że dwa elementy są w relacji {\color{red}zwrotnej}, {\color{green}przechodniej} lub {\color{blue}$\rho$}. 
	Cała tabel przedstawia natomiast relację $\rho'$ taką że $\forall m,n \in \mathbb{N} \quad m \rho' n \equiv m \leqslant n$,  będącą domknięciem relacji $\rho$ ze względu na 
	$\mathcal{R} = \{\text{zwrotność, przechodniość}\}$
\end{przyklad}

\subsection{Słowa i alfabety}
	\begin{df}
		Dowolny skończony ciąg nad danym alfabetem nazywamy słowem. $\varepsilon$ -- słowo puste, $\Sigma^*$ -- zbiór wszystkich słów
	\end{df}
	
	\begin{df}
		Dowolny podzbiór zbioru słów jest językiem $L \subset \Sigma^*$
	\end{df}

	\begin{przyklad}~\\
		\begin{itemize}
			\item $\Sigma = \{0, 1, 2, 3, 4, 5, 6, 7, 8, 9\}$
			\item $\Sigma^* = \{\varepsilon, 0, 1, \dots, 9, 00, 01, \dots, 09, 10, 11, \dots, 99, 100, \dots\}$
			\item $L = \{0, 1, 2, \dots, 9, 10, \dots, 19, 20, 21, \dots, 99, 100, \dots \}$
			\item $L' = \{2, 3, 5, 7, 11, \dots \}$
		\end{itemize}
	\end{przyklad}

	\begin{df}
		Jeżeli nad zbiorem $X$ zdefiniowano relację równoważności $\sim$ to klasą abstrakcji elementu $x\in X$ nazwiemy zbiór wszystkich elementów 
		z $X$ które są w relacji z $x$: $[x]_\sim = \{y\in X: y\sim x\}$
	\end{df}		
	
	\begin{przyklad}
		$\Sigma = \{0, 1, \dots, 9\}$, $\rho \subset \Sigma^* \times \Sigma^* \quad \forall n,m\in \Sigma^* \quad m\rho n \equiv
		 \text{wartość } m = \text{wartość } n$. Klasy abstrakcji: \\
		\begin{itemize}
			\item $A_\epsilon = \{\epsilon\}$
			\item $A_0 = \{0, 00, 000, \dots\}$
			\item $A_1 = \{1, 01, 001, \dots\}$
			\item $A_2 = \{2, 02, 002, \dots\}$
		\end{itemize}
	\end{przyklad}		
	
	\begin{przyklad}
		$\rho \subset \mathbb{N} \times \mathbb{N}:  m\rho n \equiv |m-n| = 3$ -- nie jest zwrotna ani przechodnia ale jest symetryczna.
		Domknięcie zwrotne: $m\rho n \equiv (m-n) mod 3 = 0 \vee m = \epsilon = n$. Klasy tego domknięcia abstrakcji $[\epsilon]$, $[0]$, $[1]$, $[2]$
	\end{przyklad}
	
	\begin{df}
		Powiemy że relacja $\rho$ jest prawostronnie niezmienna $\Leftrightarrow$ gdy dla dowolnych dwóch słów będących w relacji, po dopisaniu
		tego samego słowa nadal pozostaną w niej. 
		$$ \forall u, v \in \Sigma^* \quad u\rho v \Rightarrow \forall z\in\Sigma^* \quad uz \rho vz $$
	\end{df}
	
	\begin{df}
		Powiemy żę $R_L$ jest relacją indukowaną przez język 
		$L \Leftrightarrow$ $$ \forall u,v \in \Sigma^* \quad uR_Lv \equiv \forall z\in \Sigma^* \quad uz\in L \equiv vz\in L$$
	\end{df}		
	
\end{document}
