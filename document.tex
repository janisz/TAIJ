\documentclass[12pt,a4paper]{article}

\usepackage[utf8]{inputenc}
\usepackage[T1]{fontenc}
\usepackage[a4paper]{geometry}

\usepackage{pdflscape}
\usepackage{float}

\usepackage{color}

\usepackage{amsthm}
\usepackage{amsfonts}
\usepackage{amsmath} %pakiet potrzebny do \eqref
\usepackage{amssymb} %pakiet potrzebny do \blacksquare i \mathbb{•}

\usepackage[polish]{babel}

\usepackage{graphicx}

\usepackage[normalem]{ulem}

\usepackage{listliketab}
\usepackage{paralist}

\newtheorem{lemat}{Lemat}
\newtheorem{tw}{Twierdzenie}
\newtheorem{przyklad}{Przykład}
\newtheorem{zad}{Zadanie}
\theoremstyle{definition}
\newtheorem{df}{Definicja}
\newtheorem{alg}{Algorytm}
\theoremstyle{remark}
\newtheorem{uwaga}{Uwaga}

\usepackage[pdfauthor={Janisz},%
pdftitle={Teoria Automatów i Języków},%
pagebackref=true,%
pdftex]{hyperref}
\hypersetup{colorlinks=false}


\pagestyle{plain}
\begin{document}
\title{ Teoria Automatów i Języków}
\author{\vspace{-5ex}}
\date{\today}
\maketitle
\tableofcontents

\DeclareGraphicsExtensions{.pdf,.png,.jpg}
\begin{center}
\leavevmode

\vfill

\includegraphics[width=1 in]{by-sa.png}
\end{center}
\label{fig:cc}
\scriptsize{Ten utwór jest dostępny na licencji  
\href{http://creativecommons.org/licenses/by-sa/3.0/pl/}{Creative Commons Uznanie autorstwa-Na tych samych warunkach 3.0 Polska.}}

\pagebreak

\section{Relacje}

\begin{df}
Niech $X_1, X_2, \dots, X_n$ będą zbiorami. Podzbiór \href{http://pl.wikipedia.org/wiki/Iloczyn_kartezja%C5%84ski}{iloczynu kartezjańskiego} 
$R \subset X_1 \times X_2 \times \dots \times X_n$ 
nazywamy n-argumentową \href{http://pl.wikipedia.org/wiki/Relacja_(matematyka)}{relacją}. 
\end{df}

\begin{df}
Niech $X, Y$ będą zbiorami. Wtedy \href{http://pl.wikipedia.org/wiki/Relacja_dwuargumentowa}{relację dwuargumentową} nazywamy $\rho = X \times Y$, a zbiory $X, Y$ odpowiednio dziedziną i przeciwdziedziną.  
\end{df}

\begin{df}
Relacją binarną nazywamy taką relację dwuargumentową,
w której dziedzina i przeciwdziedzina są równe. $\rho = X \times X$
\end{df}

\subsection{Własności relacji}

$\forall x,y,z \in X$
\begin{listliketab}
\storestyleof{itemize} 
\begin{tabular}{ll}
	\textbullet  zwrotna & $x \rho x$ \\
	\textbullet  przeciw-zwrotna & $\neg x\rho x$ \\
	\textbullet  symetryczna & $x\rho y \Rightarrow y\rho x$ \\
	\textbullet  przeciw-symetryczna & $x \rho y \Rightarrow \neg y\rho x $ \\
	\textbullet  antysymetryczna & $x\rho y \wedge y\rho x \Rightarrow x= z$ \\
	\textbullet  przechodnia & $x\rho y \wedge y\rho z \Rightarrow x\rho z$ \\
	\textbullet  spójna & $ x\rho y \vee y\rho x $ \\
\end{tabular}
\end{listliketab}

\begin{df}
Relację nazywamy \href{http://pl.wikipedia.org/wiki/Relacja_r%C3%B3wnowa%C5%BCno%C5%9Bci}{relacją równoważności}
jeśli jest jednocześnie zwrotna, symetryczna i przechodnia
\end{df}

\begin{df}
Niech $\rho\subset X\times X$ będzie relacją binarną, a $\mathcal{R}$ zbiorem własności relacji. Powiemy że $\rho'$ jest domknięciem relacji $\rho$
ze względu na $\mathcal{R} \quad \Leftrightarrow$
\begin{enumerate}
	\item $\rho \subset \rho'$
	\item $\rho'$ jest domknięta ze względu na własności z $\mathcal{R}$
	\item $\rho'$ jest najmniejszą relacją spełniającą powyższe warunki
\end{enumerate}
\end{df}

\begin{przyklad}
	$\rho \subset \mathbb{N}\times\mathbb{N} \quad m,n \in \mathbb{N} \quad m\rho n \equiv m +1 = n$ \\
	\begin{center}
	\begin{tabular}{ccccccccc}
		$\rho$ & 0 & 1 & 2 & 3 & 4 & 5 & 6 & \dots \\
			0  & {\color{red}1} & {\color{blue}1} & {\color{green}1}  & {\color{green}1}  & {\color{green}1}  & {\color{green}1}  & {\color{green}1}  & \dots \\
			1  & {\color{yellow}0}  & {\color{red}1}  & {\color{blue}1} & {\color{green}1}  & {\color{green}1}  & {\color{green}1}  & {\color{green}1}  & \dots \\
			2  & {\color{yellow}0}  & {\color{yellow}0}  & {\color{red}1}  & {\color{blue}1} & {\color{green}1}  &  {\color{green}1} & {\color{green}1}  & \dots \\
			3  & {\color{yellow}0}  & {\color{yellow}0}  & {\color{yellow}0}  & {\color{red}1}  & {\color{blue}1} & {\color{green}1}  & {\color{green}1}  & \dots \\
			4  & {\color{yellow}0}  & {\color{yellow}0}  & {\color{yellow}0}  & {\color{yellow}0}  & {\color{red}1}  & {\color{blue}1} & {\color{green}1}  & \dots \\
			5  & {\color{yellow}0}  & {\color{yellow}0}  & {\color{yellow}0}  & {\color{yellow}0}  & {\color{yellow}0}  & {\color{red}1}  & {\color{blue}1} & \dots \\
	     \dots & \dots  & \dots  & \dots  & \dots  & \dots  & \dots  & {\color{red}1}  & \dots \\
	\end{tabular}\\
	\end{center}	
	{\color{yellow} $0$ oznacza brak relacji}, $1$ oznacza że dwa elementy są w relacji {\color{red}zwrotnej}, {\color{green}przechodniej} lub {\color{blue}$\rho$}. 
	Cała tabel przedstawia natomiast relację $\rho'$ taką że $\forall m,n \in \mathbb{N} \quad m \rho' n \equiv m \leqslant n$,  będącą domknięciem relacji $\rho$ ze względu na 
	$\mathcal{R} = \{\text{zwrotność, przechodniość}\}$
\end{przyklad}

\subsection{Słowa i alfabety}
	\begin{df}
		Dowolny skończony ciąg nad danym alfabetem nazywamy \href{http://pl.wikipedia.org/wiki/S%C5%82owo_(matematyka)}{słowem}.
		 $\varepsilon$ -- słowo puste, $\Sigma^\star$ -- zbiór wszystkich słów
	\end{df}
	
	\begin{df}
		Dowolny podzbiór zbioru słów jest językiem $L \subset \Sigma^\star$
	\end{df}

	\begin{przyklad}~\\
		\begin{itemize}
			\item $\Sigma = \{0, 1, 2, 3, 4, 5, 6, 7, 8, 9\}$
			\item $\Sigma^\star = \{\varepsilon, 0, 1, \dots, 9, 00, 01, \dots, 09, 10, 11, \dots, 99, 100, \dots\}$
			\item $L = \{0, 1, 2, \dots, 9, 10, \dots, 19, 20, 21, \dots, 99, 100, \dots \}$
			\item $L' = \{2, 3, 5, 7, 11, \dots \}$
		\end{itemize}
	\end{przyklad}

	\begin{df}
		Jeżeli nad zbiorem $X$ zdefiniowano relację równoważności $\sim$ to klasą abstrakcji elementu $x\in X$ nazwiemy zbiór wszystkich elementów 
		z $X$ które są w relacji z $x$: $[x]_\sim = \{y\in X: y\sim x\}$
	\end{df}		
	
	\begin{przyklad}
		$\Sigma = \{0, 1, \dots, 9\}$, $\rho \subset \Sigma^\star \times \Sigma^\star \quad \forall n,m\in \Sigma^\star \quad m\rho n \equiv
		 \text{wartość } m = \text{wartość } n$. Klasy abstrakcji: \\
		\begin{itemize}
			\item $A_\epsilon = \{\epsilon\}$
			\item $A_0 = \{0, 00, 000, \dots\}$
			\item $A_1 = \{1, 01, 001, \dots\}$
			\item $A_2 = \{2, 02, 002, \dots\}$
		\end{itemize}
	\end{przyklad}		
	
	\begin{przyklad}
		$\rho \subset \mathbb{N} \times \mathbb{N}:  m\rho n \equiv |m-n| = 3$ -- nie jest zwrotna ani przechodnia ale jest symetryczna.
		Domknięcie zwrotne: $m\rho n \equiv (m-n) mod 3 = 0 \vee m = \epsilon = n$. Klasy tego domknięcia abstrakcji $[\epsilon]$, $[0]$, $[1]$, $[2]$
	\end{przyklad}
	
	\begin{df}
		Powiemy że relacja $\rho$ jest prawostronnie niezmienna $\Leftrightarrow$ gdy dla dowolnych dwóch słów będących w relacji, po dopisaniu do obu tego samego słowa ze zbiru słów nadal pozostaną w relacji. 
		$$ (\forall u, v \in \Sigma^\star) \quad[u \rho v \Rightarrow (\forall z\in\Sigma^\star) uz \rho vz] $$
	\end{df}
	
\subsection{Relacje indukowane przez język}	
	
	\begin{df}
		Powiemy że $R_L$ jest relacją indukowaną przez język 
		$L \Leftrightarrow$ $$ (\forall u,v \in \Sigma^\star) \quad \{uR_Lv \equiv [(\forall z\in \Sigma^\star) \quad uz\in L \equiv vz\in L]\}$$
	\end{df}	
	
	\begin{zad} 
		Udowodnij, że relacja indukowana przez język $(R_L)$ jest relacją równoważności.
	\end{zad}
	
%Wykład 2 
%12.10.2012

	\begin{przyklad}
		Niech $L \subset \Sigma^\star$ -- język, $R_L \subset \Sigma^\star \times \Sigma^\star$\footnote{relacja jest nad zbiorem wszystkich słów z alfabetu,
		a nie nad językiem dlatego przy wyznaczaniu klas abstrakcji należy zbadać również elementy nie należące do języka}
		$R_L$ jest relacją określoną przez język $L$ w następując sposób $$L \equiv (\forall u,v \in \Sigma^\star) \quad uR_Lv \equiv [(\forall z\in \Sigma^\star) ~
		uz \in L \equiv vz \in L ]  $$
		Niech alfabet będzie alfabetem binarnym $(\Sigma = (0, 1)$, a język $L$ językiem binarnym bez znaczących zer -- $L = \{0, 1, 10, 11, 100, \dots \}$
		\footnote{słowo puste nie należy do języka}. Wtedy relacja $R_L$ tworzy następujące klasy abstrakcji:
		\begin{enumerate}
			\item $A_\epsilon = \{\epsilon\}$ -- klasa zawierająca tylko słowo puste $(\epsilon)$ 
			\item $A_0 = \{0\}$ -- klasa zawierająca tylko $0$
				\begin{itemize}
					\item $0$ jest w relacji z samym sobą
					\item nie jest w relacji z żadnym innym słowem z poza języka ponieważ:
					\begin{proof}
						Niech $z = \epsilon$ wówczas $0z = 0\epsilon = 0 \in L$ oraz $uz = u\epsilon = u \not\in L$. Zatem nie może być w relacji z żadnym 
						słowem z poza języka
					\end{proof}
					\item nie jest w relacji z żadnym słowem z języka bo:
					\begin{proof}
						Niech $z = 1$ wówczas $0z = 01 \not\in L$ oraz $uz = u1 = 1\dots 1 \in L$
					\end{proof}
				\end{itemize}
			\item $A_{10} = L - \{0\}$ -- klasa zawierająca wszystkie słowa z języka poza $0$
				\begin{itemize}
					\item Każdy element jest w relacji z elementem z klasy
					\begin{proof}
						Niech $u, v \in A_{10}$ wówczas $u = 1\dots$ i $v = 1\dots$. Dla $z \in \Sigma^\star \quad uz = 1\dots$ i $vz = 1\dots$
						stąd $uz \in L$ i $vz \in L$
					\end{proof}
					\item Każdy element nie jest w relacji z elementem nie należącym do $A_{10}$
					\begin{proof}
						Niech $u \in A_{10}$ i $v \not\in A_{10}$ wówczas $u = 1\dots$ 
						\begin{itemize}
							\item jeśli $v\ \neq \epsilon$ to znaczy $v = 0\dots $, wówczas dla $z = 1 \quad uz \in L \wedge vz \not\in L$
							\item jeśli $v = \epsilon$ to dla $z = \epsilon \quad uz \in L \wedge vz \not\in L$
							\footnote{Można skorzystać z tego że już udowodniliśmy że $\epsilon$ jest w innej klasie abstrakcji}
						\end{itemize}
					\end{proof}
				\end{itemize}
			\item $A_{01} = \Sigma^\star - (L \cup {\epsilon}) = \{ \text{słowa z wiodącymi nieznaczącymi zerami} \}$
		\end{enumerate}
		W przypadku alfabetu złożonego z większej liczby znaków dla tej relacji $R_L$, klasy abstrakcji byłyby identyczne.
	\end{przyklad}	
	
	\begin{przyklad}
		$L = $ zbiór słów takich że kolejne trójki liczb składają się z identycznych liter $ = \{\epsilon, 000, 111, 000111, 000000, \dots\}\\
		L \subset \{0, 1\}^\star$\\
		Klasy abstrakcji relacji indukowanej przez język $L$
		\begin{enumerate}
			\item $A_L = \{L\}$ -- wszystkie słowa z języka
			\item $A_0 = \{u0: ~ u\in L\}$ -- słowa z języka z dodatkowym $0$ na końcu
			\item $A_1 = \{u1: ~ u\in L\}$ -- słowa z języka z dodatkowym $1$ na końcu
			\item $A_{00} = \{u00: ~ u\in L\}$ -- słowa z języka z dodatkowym $00$ na końcu
			\item $A_{11} = \{u11: ~ u\in L\}$ -- słowa z języka z dodatkowym $11$ na końcu
			\item $A_\sim$ -- wszystkie pozostałe słowa
		\end{enumerate}
	\end{przyklad}		
	
	\begin{przyklad}
		$L = $ zbiór słów które mają tyle samo zer i jedynek $ = \{\epsilon, 01, 10, 0011, 1010, \dots\}\\
		L \subset \{0, 1\}^\star$\\
		Klasy abstrakcji relacji indukowanej przez język $L$\footnote{indeks dolny przy klasie abstrakcji to różnica pomiędzy liczą zer i jedynek}
		\begin{enumerate}
			\item $A_0 = L$
			\item $A_1 = \{0, \dots \}$
			\item $A_-1 = \{1, \dots \}$
			\item $A_2 = \{11, \dots \}$\\
			\vdots
		\end{enumerate}
	\end{przyklad}
	\begin{zad}
		Podaj klasy abstrakcji dla relacji indukowanej przez język palindromów nad alfabetem binarnym
	\end{zad}

\section{Grafy}
	
	\begin{df}
		\href{http://pl.wikipedia.org/wiki/Graf_(matematyka)}{Grafem} nazywamy parę $G = (V, E)$ gdzie $V$ to zbiór wierzchołków 
		$E \subset V \times V$ to zbiór krawędzi.
	\end{df}	

	\begin{df}
		\href{http://pl.wikipedia.org/wiki/Drzewo_(matematyka)}{Drzewo} to graf spójny graf acykliczny o następujących własnościach:
		\begin{enumerate}
			\item Jednowierzchołkowy graf $G = (\{v\}, \emptyset)$ nazywamy korzeniem drzewa
			\item Jeśli $T_1 = (v_1, E_1), ~ T_2 = (v_2, E_2), ~\dots ,~ T_k = (v_k, E_k)$ są drzewami o korzeniach $v_{01}, v_{02}, \dots, v_{0k}$
			to $$T = \underbrace{ \left(\{v_0\} \cup \bigcup^k_{i=1}V_i \right. }_\text{wierzchołki} ~,~
					\underbrace{ \left. \bigcup_{i=0}^k \{v_0, v_{0i}\} \cup \bigcup_{i=0}^k E_i \right) }_\text{krawędzie}$$
			\item Dowolną konstrukcję otrzymaną przez zastosowanie reguł 1 i 2
		\end{enumerate}			
	\end{df}
	
	\begin{df}
		Jeśli wysokościami drzew $T_1 = (V_1, E_1), ~ T_2 = (V_2, E_2), ~\dots ,~ T_k = (V_k, E_k)$ są $h_1, h_2, \dots , h_k$ to
		wysokość drzewa $T$ wynosi $h = max\{h_1, h_2, \dots , h_k\}$
	\end{df}
	
	\begin{df}
		K-drzewo to drzewo którego dowolny wierzchołek ma k następników.
	\end{df}
	
	\begin{df}[Zasada indukcji matematycznej]
		Niech $W$ będzie pewną własnością liczb naturalnych taką że:
		\begin{enumerate}
			\item $W(0)$ -- własność $W$ zachodzi dla $0$
			\item $(\forall k=0, 1, \dots) \quad W(k) \Rightarrow W(k+1)$ 
		\end{enumerate}
		Wówczas $\forall n \in \mathbb{N} \quad W(n)$ --własność $W$ zachodzi dla n
	\end{df}
	
	\begin{lemat}
		Dowolne k-drzewo o wysokości $h$ ma nie więcej niż $k^h$ liści
		\begin{proof} (\emph{indukcyjny})\\
			\begin{enumerate}
				\item dla drzewa jednowierzchołkowego liczba liści wynosi $1$, a wysokość $0$
				\item niech $T_1, T_2, \dots, T_l$ -- k-drzewa o wysokościach $h_1, h_2, \dots, h_l$
				Niech $T$ -- drzewo zbudowane z $T_1, T_2, \dots, T_l$ według reguły z definicji drzewa. 
				Wysokość $T$ jest równa $1 + max\{h_1, h_2, \dots, h_l\}$.\\
				Zakładam, że liczby liści w drzewach $T_1, \dots, T_l$ są nie większe niż $k^{h_1}, k^{h_2}, \dots, k^{h_l}$. 
				Liczba liści w drzewie $T$ jest nie większa niż 
				$$k^{h_1} + k^{h_2} + \dots + k^{h_l} \leqslant k\cdot max\{k^{h_1}, k^{h_2}, \dots, k^{h_l}\} = k\cdot k^{max\{h_1, h_2, \dots, h_l\}} 
				= k^{1 + max\{h_1, h_2, \dots, h_l\}} = k^h$$
				\item Na mocy spełnienia punktu 1 i 2 zasady indukcji matematycznej wnioskujemy iż lemat jest prawdziwy.
			\end{enumerate}
		\end{proof}
	\end{lemat}
	
\section{Wyrażenia regularne}
	
	\begin{df}(\href{http://pl.wikipedia.org/wiki/Domkni%C4%99cie_Kleene'ego}{Domknięcie Kleene'ego})
	Jeżeli $L$ jest językiem nad pewnym alfabetem to:
		\begin{itemize}
			\item $L^0 = \{\epsilon\}$
			\item $L^{n+1} = L^n \circ L$
			\item $L^* = \bigcup\limits_{k=0}^\infty L^k$
		\end{itemize}
	\end{df}	
	
	\begin{przyklad}
		Niech $L = \{00, 000\}$ będzie alfabetem, wtedy:\\
		\begin{inparaenum}[\itshape a\upshape)] 
			\item $L^0 = \{\epsilon\}$
			\item $L^1 = \{00, 000\}$
			\item $L^2 = \{0000, 00000, 000000\}$
			\item $L^* = \{\epsilon, 00, 000, 0000, 00000, \dots\}$
		\end{inparaenum}
	\end{przyklad}
	
	\begin{df} \href{http://pl.wikipedia.org/wiki/Wyra%C5%BCenia_regularne}{Wyrażeniem regularnym} nad alfabetem $\Sigma$ nazywamy konstrukcje
	 powstałe w wyniku skończonej iteracji poniższych zasad.
		\begin{enumerate}
			\item Podstawowe wyrażenia regularne: \footnote{Poniższe znaki powinny zostać wyróżnione, aby zaznaczyć używanie ich w kontekście
			wyrażeń regularnych. Jednak ponieważ zazwyczaj ich znaczenie wynika wprost z kontekstu, nie zastosowano tu żadnego wyróżnienia}
				\begin{itemize}
					\item $\emptyset$ -- brak znaku
					\item $\epsilon$ -- słowo puste
					\item $\Sigma$ -- alfabet $\forall a\in\Sigma ~~ a$
				\end{itemize}
			\item Jeśli $r$ i $s$ są wyrażeniami regularnymi to następujące wyrażenia też są wyrażeniami regularnymi
				\footnote{W dalszej części przyjmujemy natępujące priorytety operacji (od najniższego) suma, konkatenacja, domknięcie Kleene'ego. 
				$((a+b)(a+(b^*))) \equiv (a+b)(c+b^*)$}
				\begin{itemize}
					\item $(r+s)$ -- suma
					\item $(rs)$ -- konkatenacja
					\item $(r^*)$ -- domknięcie Kleene'ego
				\end{itemize}
		\end{enumerate}		
	\end{df}	
	
	\begin{uwaga}
	Wyrażenia regularne generują języki.
	\end{uwaga}
	
	\begin{tw}
		Jeżeli $r$ i $s$ genrują języki $R$ oraz $S$ to:
		\begin{itemize}
			\item $r+s$ generuje $R\cup S$
			\item $rs$ generuje $RS = \{uv: ~ u\in R, ~ v\in S\}$
			\item $r^*$ generuje $R^*$
		\end{itemize}
	\end{tw}
	
	\begin{df}
		Językami regularnymi nazywamy takie i tylko takie języki, które zostały wygenerowane przez wyrażenie regularne
	\end{df}	
	
	\begin{przyklad}
		Sprawdzić czy Język liczb dziesiętnych bez nieznaczących zer jest językiem regularnym. \\
		\emph{Odp.} Tak, generuje go następujące wyrażenie regularne:
		$$  0 + (1+2+\dots+9)(0+1+2+\dots+9)^* $$
		\begin{proof}
			Należy wykazać że każde słowo generowane przez to wyrażenie jest liczbą dziesiętną, bez nieznaczących zer oraz każda liczba 
			dziesiętna bez nieznaczących zer jest generowana przez to wyrażenie.\\
			Generujemy $0$ lub liczbę rozpoczynającą się do cyfry $> 0$, po której następuje dowolna ilość dowolnych cyfr.
		\end{proof}
	\end{przyklad}		
	
	\begin{przyklad}
		Wykazać że język słów binarnych w których sekwencja trzech kolejnych jedynek występuje co najmniej raz,
		jest językiem regularnym. \\
		\emph{Odp.} Tak, generuje go następujące wyrażenie regularne:
		$$[00^*](\epsilon+1+11)00]111[00^*(\epsilon+1+11)]^* $$
	\end{przyklad}
	
	\begin{zad}
		Sprawdzić czy język słów binarnych, które zaczynają i kończą się tą samą literą jest językiem regularnym.
	\end{zad}
	\begin{zad}
		Sprawdzić czy język słów binarnych, którego kolejne sekwencje tych samych liter są nie krótsze niż poprzednie jest językiem regularnym.
	\end{zad}
	
	\subsection{Tożsamości wyrażeń regularnych}
	\begin{eqnarray}
		\emptyset + r &=& r + \emptyset \\
		r + s &=& s + r \\
		(r+s) + t &=& r + (s+t) \\
		\epsilon + r &=& r \\
		(rs)t &=& r(st) \\
		r(s+t) &=& rs + rt \\
		(r^*)^* &=& r^* \\
		(r^*s^*)^* &=& (r+s)^* \\
		(\epsilon + r)^* &=& r^* \\
		(r^* + s^*)^* &=& (r + s)^*\\		
	\end{eqnarray}
	
	\begin{zad}
		Udowodnić powyższe tożsamości
	\end{zad}
	
	\begin{lemat}{\href{http://pl.wikipedia.org/wiki/Twierdzenie_Myhilla-Nerode%27a}{Myhill-Nerode}}
		Język $L$ jest regularny $\Leftrightarrow$ liczba kals abstrakcji relacji $R_L$ indukowanej przez ten język jest skończona.
	\end{lemat}	
	
	\begin{lemat}{\href{http://pl.wikipedia.org/wiki/Lemat_o_pompowaniu_dla_j%C4%99zyk%C3%B3w_regularnych}{O pompowaniu}}
		Jeśli język $L$ jest regularny to istnieje stała $n_L$ taka że dla każdego słowa $z\in L$ spełniona jest implikacja
		\footnote{jeśli słowo jest dłuższe niż $n_L$ to istnieje taki podział słowa $z$ na 3 części taki, że suma długości pierwszej
		i drugiej jest mniejsza niż $n_i$ i trzecia nie jest pusta. Wtedy słowo $z$ złożone z częśi pierwszej połączonej z powieloną 
		częścią drugą i zakończone częścią trzecią należy do $L$}
		$$|z| \geqslant n_L \Rightarrow (\exists u,v,w ~~ z = uvw ~~ |uv| \leqslant n_L, ~~ |v| \geqslant 1)
		(\forall i\in\mathbb{N}) z_i = uv^iw \in L$$
	\end{lemat}
	
	\begin{lemat}{Kontrapozycja lematu o pompowaniu}
		Jeśli dla dowolnej stałej $n$ istnieje słowo $z\in L$ takie że 
		$|z|\geqslant n$ i $(\forall u,v,z: ~~ z = uvw ~~ |uv| < n, ~ |v| \geqslant 1)(\exists i\in\mathbb{N}) ~~ z = uv^iw \not\in L$
		to język $L$ nie jest regularny.
	\end{lemat}


	\begin{przyklad}
		Sprawdzić żę język palindromów nad alfabetem binarnym nie jest językiem regularnym.
		\begin{proof}
			\begin{itemize}
				\item[sposób 1] Wystarczy wskazać nieskończony ciąg klas abstrakcji aby udowodnić że język nie jest regularny.
				Pokażę że nieskończony ciąg słów z których każde jest w innej klasie abstrakcji: 
				$z_0 = \epsilon, ~ z_1 = 0, ~ z_2 = 00, ~\dots$.\\
				Dowolne 2 słowa tego ciągu należą do różnych klas abstrakcji.
				$z_i = 0^i \quad z_j = 0^j ~~ z \neq j, ~~ z_i  \xout{R_L}z_j$ bo dla $w = 10^i ~~
				\begin{matrix}
					z_iw \in L \\ z_jw \not\in L
				\end{matrix}$
				\item[sposób 2] Niech $N$ -- stała, z kontrapozycji lematu o pompowaniu. Niech $z = 0^N10^N$, dowolny podział
				spełniający założenia tego lematu to taki którego środkowa część podziału będzie się składać
				z zer i początkowej części słowa.
				$$z = \underbrace{0^p}_u\underbrace{0^q}_v\underbrace{0^r10^N}_w\text{ , gdy }
				\left\lbrace \begin{matrix}
					p+q+r &=& N \\ p + q &\leqslant& N \\ 1 &\leqslant& q
				\end{matrix} \right.$$
				 Niech $i=0$ wówczas $z_0 = 0^p0^r10^N ~~ p+r < N \Rightarrow z_0 \not\in L$
			\end{itemize}
		\end{proof}
	\end{przyklad}	
	
	\begin{zad}
		Sprawdzić czy język słów binarnych $L = \{ w\in \{0,1\}^*: ~ \#_0w < \#_1w\}$\footnote{$\#$ oznacza liczbę wystąpień znaku z 
		indeksu dolnego w słowie}, jest regularny.
	\end{zad}
	\begin{zad}
		Sprawdzić czy język słów binarnych które składają się z sekwencji tych samych liter nie krótszych niż trzy, jest regularny.
	\end{zad}
	
\section{Gramatyki}
	
	\begin{df}
	\href{http://pl.wikipedia.org/wiki/Gramatyka_formalna}{Gramatyka} to sposób opisu języka, czyli podzbioru zbioru wszystkich
	 słów nad danym alfabetem.
	\end{df}
	\begin{df}[Relacja wywodu bezpośredniego]
	$C(V \cup T)* \times (V \cup T)*$
	\end{df}
	
	\begin{tw}
	Dwa słowa są w relacji wywodu bezpośredniego gdy jedno powstało z drugiego w wyniku pewnej produkcji
	\end{tw}
	
	\begin{df}
	Relacja wywodu jest przechodnim domknięciem relacji wywodu bezpośredniego.
	\end{df}
	
	\begin{df}
	Językiem generowanym przez gramatykę $G$ nazywamy zbiór słów nad alfabetem terminali, które pozostają w relacji wywodu z 
	symbolem początkowym.
		\begin{eqnarray}
			L(G) = \{ w\in T^\star: S \Rightarrow w \}
		\end{eqnarray}
	\end{df}
	
	\begin{df}
	\href{http://pl.wikipedia.org/wiki/Gramatyka_regularna}{Gramatyki regularne} są to gramatyki prawostronnie liniowe lub 
	lewostronnie liniowe
	\end{df}
	
	\begin{df}
	Gramatykę nazywamy prawostronnie liniową $\Leftrightarrow$ wszystkie jej produkcje są postaci 
	\begin{eqnarray}
		A&\rightarrow&wB \\
		A&\rightarrow&w
	\end{eqnarray}
	gdzie $A, B \in V, ~ w\in T^\star$\footnote{początkowe wielkie litery alfabetu oznaczają symbole nieterminalne, 
	natomiast początkowe małe litery alfabetu to symbole terminalne. Greckie litery oznaczają ciągi symboli terminalnych i nieterminalnych
	Końcowe litery alfabetu oznaczają ciągi terminali}	 
	\end{df}
	
	\begin{przyklad}
		Konstrukcja gramatyki generującej język liczb binarnych, bez nieznaczących zer \\
		$G = (V, P, T, S) \quad V = \{S\} \cup \{A\} \quad T = \{0, 1\}$\footnote{Jeśli nie określono inaczej to 
		za symbol początkowy przyjmujemy $S$}
		$$
		\begin{matrix}
		S \to 0 & S \to 1 & S \to 1A \\
		A \to 0A & A \to 1A & A \to 1 & A \to 0\\		
		\end{matrix}
		\Rightarrow
		S \to 0|1|1|A, \quad A \to 0|1|0A|1A
		$$
	\end{przyklad}		
	
	\begin{tw}
		Gramatyki regularne generują tylko języki regularne
	\end{tw}
		
\subsection{\href{http://pl.wikipedia.org/wiki/Gramatyka_bezkontekstowa}{Gramatyki bezkontekstowe}}
	
	\begin{df}[Gramatyka bezkontekstowa]~\\
		$G$ jest gramatyką bezkontaktową $\Leftrightarrow$ lewa strona każdej produkcji gramatyki $G$ jest złożona z jednego symbolu
		nieterminalnego to znaczy ma produkcje postaci:
			$$
				(A, \alpha), \text{gdzie} A\in V, \alpha \in \{T,V\}\star
			$$
	\end{df}
	
	\begin{df}[Język bezkontekstowy]~\\
		Językami bezkontekstowymi będziemy nazywać języki generowane przez gramatyki bezkontekstowe i tylko takie języki		
		
		\begin{uwaga}
			Językiem generowanym przez gramatykę bezkontekstową jest zbiór słów który da się wywieść z symbolu początkowego gramatyki stosując
			jej produkcje
		\end{uwaga}
	\end{df}
	
	\begin{df}[Drzewo wywodu w gramatyce bezkontekstowej]~\\
		Jest to drzewo o następujących własnościach
		\begin{itemize}
			\item Wierzchołki etykietowane symbolami terminalnymi, nieterminalnymi lub $\varepsilon$
			\item Korzeń etykietowany symbolem początkowym gramatyki
			\item Dla produkcji $A\rightarrow x_1x_2 \dots x_n$, w drzewie wierzchołek $A$ ma następniki 
			kolejno $x_1, x_2, ..., x_n$ i odwrotnie
			\item Jeżeli wierzchołek etykietowany jest słowem pustym ($\varepsilon$) to jest jedynym
			następnikiem swojego przodka
		\end{itemize}
	\end{df}

	
	\begin{przyklad}
		Wykazać że język palindromów jest generowany przez gramatykę bezkontekstową.
	\begin{proof}
		$w$ - palindrom nad $\{0,1\}$ \\
		\begin{enumerate}
			\item Jeśli $w$ jest słowem pustym ($\epsilon$) lub jedną literą to jest generowane
			\item Wykażemy że jest palindrom długości nie większych niż $n$ są generowane w tej gramatyce to
			generowane są również palindromy od długości $n+1$ i $n+2$\\
		\end{enumerate}\footnote{ten dowód można przeprowadzić rozpatrując parzystość $n$}
	\end{proof}
	\end{przyklad}		
	
	\begin{tw}
	Klasa języków regularnych zawiera się w klasie języków bezkontekstowych
	\end{tw}
	
	\subsubsection{Metody upraszczania gramatyk bezkontekstowych}
	
	\begin{df}[Symbole bezużyteczne]~\\
		\begin{description}
			\item[niegenerujące] symbole nieterminalne z których nie da się wywieść ciągu symboli terminalnych (być może pustego)
			\item[nieosiągalne] symbole których nie da się wywieść z symbolu początkowego
		\end{description}				
	\end{df}	
	
	\begin{lemat}[Usuwanie symboli bezużytecznych 1]\footnote{Uzasadnienie lematu można przeprowadzić analizując drzewa wywodu gramatyk}~\\
		Dla dowolnej gramatyki bezkontekstowej $G = (V, T, P, S)$ generującej niepusty język $L(G)$ istnieje równoważna gramatyka
		bezkontekstowa $G' = (V', T', P', S')$ tak że z każdego symbolu nieterminalnego z $V'$ można wyprowadzić ciąg symboli terminalnych (być może pusty)
	\end{lemat}	
	
	\begin{lemat}[Usuwanie symboli bezużytecznych 2]\footnote{Uzasadnienie lematu można przeprowadzić analizując drzewa wywodu gramatyk}~\\
		Dla dowolnej gramatyki bezkontekstowej $G = (V, T, P, S)$ generującej niepusty język $L(G)$ istnieje równoważna gramatyka
		bezkontekstowa $G' = (V', T', P', S')$ tak że każdy symbol terminalny i nieterminalny z $T' \cup V'$ można wyprowadzić
		z  symbolu startowego $S$
	\end{lemat}		
	
	\begin{alg}[Usuwanie symboli niegenerujących]~\\
		\begin{enumerate}
			\item Wszystkie terminale oznaczamy jako generujące
			\item Symbole (nieterminalne) mające po prawej stronie produkcji symbol/e generujące oznaczamy jako generujące
			\item Powtarzamy punkt 2 dopóki zbiór symboli generujących nie będzie stabilny
			\item Usuwamy wszystkie symbole nie oznaczone jako generujące oraz produkcje które je generują
		\end{enumerate}
	\end{alg}
	
	\begin{alg}[Usuwanie symboli nieosiągalnych]~\\
		\begin{enumerate}
			\item Oznaczamy symbol początkowy gramatyki i symbole przez niego generowane (po prawej stronie produkcji)
			jako osiągalne
			\item Symbole generowane z symboli osiągalnych oznaczamy jako osiągalne
			\item Powtarzamy punk 2 dopóki zbiór symboli osiągalnych nie będzie stabilny
			\item Usuwamy wszystkie symbole nie oznaczone jako osiągalne oraz produkcje które je zawierają
		\end{enumerate}
	\end{alg}	
	
	\begin{tw}
		Po zastosowani lematów w podanej kolejności uzyskamy gramatykę bez symboli bezużytecznych
	\end{tw}
	
	\begin{df}
		Symbol nieterminalny nazywamy wycieralnym $\Leftrightarrow$ można z niego wywieść ciąg pusty
	\end{df}		

	\begin{df}
		Produkcję nazywamy wycieralną jeśli prawą strona tej produkcji można przekształcić do symbolu pustego
	\end{df}			
	
	\begin{alg}[Metoda wyznaczania symboli wycieralnych]\footnote{Uzasadnienie poprawności tej metody polega na 
	obserwacji własności wywodu słów w obu gramatykach -- przed i po usuwaniu}~\\
		\begin{enumerate}
			\item Przeglądamy produkcję i za symbole wycieralne uznajemy te symbole które po lewej stronie produkcji 
			prowadzą w słowo puste
			\item Dopóki zbiór symboli wycieralnych się zmienia
			\begin{enumerate}
				\item przeglądamy wszystkie produkcje i za symbole wycieralne uznajemy te które po są po lewej stronie
				produkcji a po prawej ciągi symboli już uznanych za wycieralne
			\end{enumerate}
		\end{enumerate}
	\end{alg}
	
	\begin{df}[Usuwanie symboli bezużytecznych i wycieralnych]~\\
		Jeśli $G = (V, T, P, S)$ jest gramatyką bezkontekstową i bez symboli bezużytecznych i nie generuje słowa pustego
		to można znaleźć równoważną jej gramatykę bez symboli bezużytecznych i wycieralnych.\\
		\emph{Metoda przekształcania}\\
		Załóżmy że $A \to X_1,X_2,\dots,X_n$ jeśli któryś z symboli $X_i$ jest wycieralny to tę produkcję zastępujemy produkcjami
		$\begin{matrix}
			A \to& X_1,X_2,\dots,X_{i-1},X_i,X_{i+1},X_n \\ A \to& X_1,X_2,\dots,X_{i-1},X_{i+1},X_n
		\end{matrix}$
		z gramatyki zawsze usuwamy produkcje w słowo puste
	\end{df}	
	
	\begin{lemat}[Produkcje jednostkowe]~\\
		Niech $G$ będzie gramatyką bezkontekstową bez symboli bezużytecznych oraz wycieralnych. Wówczas 
		istnieje równoważna jej gramatyka $G'$ również bez produkcji jednostkowych
	\end{lemat}
	
	\begin{alg}[Usuwanie produkcji jednostkowych]\footnote{Uzasadnienie jest analogiczne do algorytmu usuwania symboli wycieralnych}~\\
		\begin{enumerate}
			\item Usuwamy wszystkie produkcje w ten sam nie terminal
			\item Produkcje z nieterminalna $A$ w inny nie terminal $B$ zamieniamy na produkcję z $A$ w prawą stronę produkcji z $B$
		\end{enumerate}
	\end{alg}
	
\subsection{Postaci gramatyk bezkontekstowych}

	\begin{tw}
		Jeśli $L$ jest językiem bezkontekstowym, to język $L' = L - \varepsilon$ też jest językiem bezkontekstowym
	\end{tw}
	
	\begin{df}[\href{http://pl.wikipedia.org/wiki/Posta\%C4\%87_normalna_Chomsky\%27ego}{Postać normalna Chomskiego}]~\\
		Gramatyka $G$ jest w postaci normalnej Chomskiego $\Leftrightarrow$ gdy wszystkiego jej produkcje są postaci:
		\begin{itemize}
			\item $A\rightarrow BC$
			\item $A \rightarrow \alpha$
		\end{itemize}
		gdzie $A,B,C \in V$ (symbole nieterminalne), $\alpha \in T$ (symbol terminalny)
	\end{df}
	
	\begin{lemat}
		Każdą gramatykę bezkontekstową, która nie generuje słowa pustego można przekształcić do postaci Chomskiego
	\end{lemat}
	
	\begin{alg}[Przekształcanie do postaci Chomskiego]~\\
		\begin{enumerate}
			\item Usuwamy z gramatyki symbole bezużyteczne, wycieralne i produkcje jednostkowe
			\item Jeżeli po prawej stronie występuje więcej niż jeden (1) symbol, to wszystkie
			terminale tam występujące zamieniamy na nowe nie-terminale i dodajemy
			nowe produkcje z nowych nie-terminali w odpowiadające im terminale
			\item Wszytki pozostałe produkcje w ciąg nie-terminali długości większej niż dwa (2)
			modyfikujemy w sposób następujący
				\begin{enumerate}
					\item Dopóki ciąg nie-terminali po prawej stronie produkcji nie zmniejszy się do długości dwa (2)
					zastępujemy jakąś parę nie-terminali z tego ciągu nowym nie-terminalem i dodajemy nową produkcję
					z nowego nie-terminala w dwa stare, usunięte nie-terminale
				\end{enumerate}
		\end{enumerate}
	\end{alg}
	
	\begin{uwaga}
		Drzewo wywodu słowa w gramatyce bezkontekstowej w postaci Chomskiego jest drzewem binarnym
	\end{uwaga}
	
	\begin{df}[\href{http://pl.wikipedia.org/wiki/Posta\%C4\%87_normalna_Greibach}{Postać normalna Greibach}]~\\
		Pramatka bezkontekstowa $G$ jest w postaci Greibach $\Leftrightarrow$ wszystkie jej produkcje są postaci:
		\begin{equation}
			A \rightarrow a\alpha
		\end{equation}
		gdzie $a\in T$, $\alpha \in V\star$ (ciąg nie-terminali)
	\end{df}
	
	\begin{lemat}
		Każdą gramatykę bezkontekstową nie generującą słowa pustego da się sprowadzić do postaci Greibach
	\end{lemat}
	
	\begin{alg}[Przekształcanie do postaci Greibach]~\\
		\begin{enumerate}
			\item Sprowadzamy gramatykę do postaci Chomskiego
			\item Numerujemy symbole nieterminalne (ustawiając je w ciąg $V = \{A_1, A_2, \dots, A_n\}$
			\item Otrzymane $A_i$ produkcje spełniają warunek, że dla $i=1,\dots, k-1$ produkcje z $A_i$ są postaci
				\begin{itemize}\label{warunek}
					\item $A_i \rightarrow a\alpha$, gdzie $a\in T$ i $\alpha\in V\star$
					\item $A_i \rightarrow A_j\alpha$, gdzie $j > i$ 
					oraz $A_j$ jest nie-terminalem z naszego ciągu o indeksie $j$ i $\alpha \in V\star$
				\end{itemize}
			Ponadto $A_k$ nie spełnia tego warunku wiec mamy dwa przypadki
				\begin{enumerate}
					\item $A_k \rightarrow A_j\alpha$ oraz $j < k$\label{1}
					\item $A_k \rightarrow A_k\alpha$\label{2}
				\end{enumerate}
			\item W przypadku \ref{1} zastępujemy nie-terminale $A_j$ prawymi stronami $A_j$-produkcji
			i postępujemy tak aż do wyeliminowania problemu, po czym wracamy do punktu 3, chyba że $k = n$
			\item W przypadku gdy natrafiliśmy na sytuację \ref{2} usuwamy problem w sposób następujący:
				\begin{enumerate}
					\item Produkcję $A_k \rightarrow A_k\alpha_1|A_k\alpha_2|\dots|A_k\alpha_m|\beta_1|\dots|\beta_n$
					zastępujemy produkcjami:
					\begin{eqnarray}
						A_k &\rightarrow& \beta_1|\dots|\beta_n|\beta_1B_k|\dots|\beta_nB_k \\
						B_k &\rightarrow& \alpha_1|\dots|\alpha_m|\alpha_1B_k|\dots|\alpha_mB_k									
					\end{eqnarray}
					gdzie $B_k$ jest nowym nie-terminalem
					\item Stosując punkt 4 i 5 zapewniliśmy spełnienie warunku \ref{warunek} przez $A_k$ i możemy zająć się produkcją $A_{k+1}$
					\item Powtarzamy czynność z punktu 3 i w miarę potrzeb 4 i 5 aż wszystkie $A_i$ nie będą spełniały \ref{warunek}
				\end{enumerate}
			\item Wszystkie produkcje $A_i \rightarrow a\alpha$ są już w postaci Greibach natomiast produkcje pozostałe które są w postaci
			$A_i\rightarrow A_j\alpha$, $j > i$ modyfikujemy zamieniając $A_j$ na prawą stronę produkcji $A_j$
		\end{enumerate}
	\end{alg}
	
\subsubsection{Sprawdzanie czy język jest bezkontekstowy}

	\begin{df}[Niejednoznaczność gramatyki]~\\
		Gramatykę nazywamy niejednoznaczną $\Leftrightarrow$ istnieje takie słowo dla którego istnieje więcej niż jedno drzewo wywodu		
	\end{df}
	
	\begin{df}[Plon drzewa]~\\
		Plonem drzewa nazywamy wszystkie słowa które można wyprowadzić z danego drzewa
	\end{df}	
	
	\begin{lemat}[\href{http://pl.wikipedia.org/wiki/Lemat_o_pompowaniu_dla_j\%C4\%99zyk\%C3\%B3w_bezkontekstowych}{O pompowaniu (dla języków bezkontekstowych)}]~\\
		Jeśli $L$ jest językiem bezkontekstowym, to istnieje stała $n_L$ taka że dla
		dowolnego słowa $z\in L$ spełniony jest warunek:
		\begin{equation}
			(|z| \geqslant n_L) \Rightarrow [( \exists_{u,v,w,x,y} z = uvwxy \wedge |vwx| \leqslant n_L \wedge |vx| \geqslant 1) \forall_{i=0,1,\dots} z_i = uv^iwx^iy \in L]
		\end{equation}
		\begin{uwaga}
			Lemat o pompowaniu dla języków regularnych jest szczególnym przypadkiem tego lematy
		\end{uwaga}
		
		\begin{proof}~\\
			\begin{enumerate}
				\item Jeżeli $L$ jest skończony, to lemat jest sprawdzimy (wystarczy przyjąć $n_L$ wiesze niż długość najdłuższego słowa)
				\item Język jest nieskończony
					\begin{enumerate}
						\item Skoro język jest bezkontekstowy to weźmy gramatykę $G$ w postaci Chomskiego generującą ten język. Wtedy 
						drzewo wywodu słów z $L$ będą drzewami binarnymi. Drzewo wywodu mające $k$ liści ma zatem $log_2k$ poziomów
						(wysokość $h = log_2k$). Zatem liczba krawędzi między ścieżkami z korzenia do liścia jest nie mniejsza niż $h$.
						Wierzchołki ścieżki oprócz ostatniego etykietujemy symbolami nieterminalnymi a ostatni symbolem terminalnym.
						\item Niech $|V| = n$ (mamy $n$ symboli nieterminalnych), wtedy przynajmniej $n_L = 2^{n+1}$, a co za tym 
						idzie drzewo wywodu słowa $z$ o długości nie mniejszej niż $n_L$ będzie miało wysokość przynajmniej $n+1$
						Wynika z tego że na ścieżce korzeń-liść pojawią się przynajmniej dwa wierzchołki etykietowane tym samym terminalem
						\item Oznaczamy przez $A_i$ najbliższe wystąpienie powtarzającego się nieterminalna licząc od liścia, a przez $A_k$
						jego drugie wystąpienie. Oznaczamy przez $w$ plon poddrzewa $A_l$ oraz przez $v$ i $x$ części plony poddrzewa $A_k$
						pozostałe po wyłączeniu w. Oznaczamy przez $u$ i $y$ części plony całego drzewa pozostałe po wyłączeniu $vwx$. 
						Poprze takie oznaczenia podzieliliśmy słowo $z$ na pięć części $z = uvwx$. Z wyboru najbliższej pary
						powtarzających się nie-terminali wynika że $|vwx| \leqslant n_L$, bo długość ścieżki jest nie większa $n+1$
						($A$ wybieramy dwukrotnie, a pozostałe nie-terminale najwyżej jednokrotnie)
						\item Dla $i=0$ z lematu równoważnym jest zastąpienie poddrzewa $A_k$ poddrzewem $A_L$ takie drzewo wywodu jest 
						nadal poprawnym drzewem wywody w gramatyce $G$, a na dodatek generuje słowo $z_0 = uwy$
						\item Ostatnią rzeczą którą trzeba pokazać to że dla $i = k \geqslant 2$  z lematu równoważnym jest $k$-krotnemu
						zastąpieniu poddrzewa $A_L$ przez poddrzewo $A_\alpha$. Znowu otrzymamy drzewo wywodu w gramatyce $G$, tym
						razem generujące słowo $z_k = uv^kwx^ky$.
					\end{enumerate}					 
			\end{enumerate}
		\end{proof}				
	\end{lemat}
			
	\begin{lemat}[Zaprzeczenie lematu o pompowaniu]~\\
		Jeśli dla dowolnej stałej $n\in\mathbb{N}$ istnieje $z\in L$ takie że
		\begin{equation}
			(|z| \geqslant n_L) \Rightarrow [( \forall_{u,v,w,x,y} z = uvwxy \wedge |vwx| \leqslant n_L \wedge |vx| \geqslant 1) \exists_{i=0,1,\dots} z_i = uv^iwx^iy \not\in L]
		\end{equation}
		to język $L$ nie jest językiem bezkontekstowym
	\end{lemat}			
		
	\begin{lemat}[\href{http://pl.wikipedia.org/wiki/Lemat_Ogdena}{Ogdena}]~\\
		Jeśli $L$ jest językiem bezkontekstowym, to istnieje stała $n_L$ taka, że dla dowolnego słowa z języka $L$, w którym wyróżniono przynajmniej
		$n_L$ symboli spełniony jest warunek:\\
		Istnieje taki podział słowa $z$ na pięć części $z = uvwxy$ takich że:
		\begin{enumerate}
			\item $vx$ zawiera przynajmniej jeden wyróżniony symbol
			\item $vwx$ zawiera co najwyżej $n_L$ wyróżnionych symboli
			\item $(\forall i\geqslant 0) \quad uv^iwx^iy \in L$
		\end{enumerate}
		
		\begin{proof}
			Dowód jest analogiczny do dowodu lematu o pompowaniu, z tym że konstrukcja ścieżki ulega drobnej modyfikacji.
			Podczas konstrukcji ścieżki wybieramy następnika którego poddrzewo o korzeniu w tym wierzchołku zawiera nie
			mniej wyróżnionych symboli, niż poddrzewo drugiego następnika
		\end{proof}			
		
		\begin{uwaga}
			Lemat o pompowaniu jest szczególnym przypadkiem lematu Ogdena
		\end{uwaga}
	\end{lemat}	
	
	\begin{lemat}[Zaprzeczenie lematu Ogdena]~\\
		Jeśli dla $\forall n\in \mathbb{N}$ istnieje słowo $z \in L$ w którym wyróżniono co najmniej $n$ symboli oraz
		dla każdego podziału słowa na pięć części takiego że
		\begin{itemize}
			\item $vx$ zawiera przynajmniej jeden wyróżniony symbol
			\item $vwx$ zawiera co najwyżej $n$ wyróżnionych symboli
		\end{itemize}
		\begin{equation}
			(\exists i=1,2,\dots) \quad uv^iwx^iy \in L
		\end{equation}
		to język $L$ nie jest językiem bezkontekstowym
	\end{lemat}	

\subsection{Gramatyki kontekstowe}

	\begin{df}[\href{http://pl.wikipedia.org/wiki/Gramatyka_kontekstowa}{Gramatyka kontekstowa}]~\\
		Gramatykę $G$ nazywamy kontekstową $\Leftrightarrow$ wszystkie produkcje gramatyki $G$ są nieskracalne
		to znaczy są postaci 
		\begin{equation}
			\alpha\rightarrow\beta
		\end{equation}
	    gdzie $\alpha,\beta\in (V\cup T)^\star$ oraz $0<|\alpha|\leqslant |\beta|$
	\end{df}			
		
	\begin{df}[Język kontekstowy]~\\
		Językami kontekstowymi nazywamy języki generowane przez gramatyki kontekstowe i tylko takie języki
	\end{df}	
	
	\begin{tw}
		Klasa języków bezkontekstowych bez słowa pustego zawiera się w klasie języków kontekstowych
	\end{tw}

	\begin{df}[Postać normalna gramatyki kontekstowej]~\\
		Postacią normalną gramatyki kontekstowej $G$ nazywamy gramatykę, której wszystkie produkcje są postaci
		\begin{equation}
			\gamma A \delta \rightarrow \gamma \alpha \delta
		\end{equation}
		gdzie $A\in V$ oraz $\alpha,\gamma,\delta\in (V\cup T)^\star \wedge \alpha \neq \varepsilon$
	\end{df}
	
	\begin{lemat}
		Każdą gramatykę kontekstową można sprowadzić do postaci normalnej
	\end{lemat}
	
	\begin{alg}[Przekształcanie do postaci normalnej]~\\
		\begin{enumerate}
			\item Przekształcamy tylko produkcje które nie są w postaci normalnej
			\item W tych produkcjach terminale zastępujemy dodatkowymi nie terminalami i dostajemy nowe produkcje
			z tych nie-terminali w zastąpione terminale
			\item Pozostałe produkcje nie będące w postaci normalnej wyglądają następująco
				\begin{equation}
					A_1A_2A_3\dots A_k \rightarrow B_1B_2\dots B_k\dots B_{k+l}, \quad l > 1, k \leqslant l 
				\end{equation}
			zastępujemy je poniższymi produkcjami:
				\begin{itemize}
					\item Tworzymy nowy symbol nieterminalny $A^r_k$
					\item Nowe produkcje
						\begin{eqnarray}						
							A_1A_2A_3\dots A_{k-1}A_k &\rightarrow& A_1\dots A_{k-1}A^r_k \\
							A_1A_2A_3\dots A_{k-1}A^r_k &\rightarrow& B_1A_2\dots A_{k-1}A^r_k \\
							B_1A_2A_3\dots A_{k-1}A^r_k &\rightarrow& B_1B_1A_3\dots A_{k-1}A^r_k\\
							&\vdots&\\
							B_1\dots B_{k-1}A^r_k &\rightarrow& B_1B_2\dots B_k\dots B_{k-l}\\
						\end{eqnarray}
				\end{itemize}
		\end{enumerate}
	\end{alg}
	
	
\section{Maszyny Turinga}
\begin{df}[Maszyna Turinga - model podstawowy]~\\
	Maszyną Turinga w modelu podstawowym nazywamy system 
	\begin{equation}
		M = (Q, \Sigma, \Gamma, \delta, q_0, B, F)
	\end{equation}
	\begin{description}
		\item[$Q$] -- skończony zbiór stanów
		\item[$\Sigma$] -- zbiór symboli wejściowych
		\item[$\Gamma$] -- skończony zbiór wszystkich symboli będący alfabetem taśmy
		\item[$\delta$] -- funkcja przejścia $\delta = Q\times\Gamma \rightarrow Q\times\Gamma\times\{L,P\}$
		\item[$q_0$] -- stan początkowy $q_0 \in Q$
		\item[$B$] -- zdefiniowany symbol pusty $B\in\Gamma$
		\item[$F$] -- zbiór stanów akceptujących $F\subset Q$
	\end{description}
	\begin{uwaga}
		Obliczenie Maszyny Turinga polega na wykonaniu kolejnych ruchów określonych funkcją przejścia
	\end{uwaga}		
	\begin{uwaga}
		Obliczenie może się zakończyć ale może też być kontynuowane w nieskończoność
	\end{uwaga}
\end{df}

\begin{df}[Relacja ruchu]
	Obliczeniem Maszyny Turinga nazywamy ciąg opisów chwilowych pozostających w relacji
\end{df}

\begin{df}[Maszyna Turinga z własnością stopu]~\\
	Maszyny Turinga które zawsze kończą obliczenie niezależnie od danych nazywamy Maszynami Turinga 
	z własnością stopu albo algorytmami
\end{df}

\begin{df}[Maszyna Turinga z własnością stopu ze stanem akceptującym]~\\
	Maszyną Turinga z własnością stopu ze stanem akceptującym nazywamy system
	\begin{equation}
		M = (Q, \Sigma, \Gamma, \delta, q_0, B, F = \{ACC\})
	\end{equation}
	Przy czym istnieje założenie, że przejście do stanu akceptującego jest zakończeniem obliczeń
\end{df}

\begin{df}[Maszyna Turinga z własnością stopu]~\\
	Maszyną Turinga z własnością stopu nazywamy system 
	\begin{equation}
		M = (Q, \Sigma, \Gamma, \delta, q_0, B, F = \{ACC\}, N = \{REJ\})
	\end{equation}
	gdzie $N$ to zbiór stanów odrzucających.
	\begin{itemize}
		\item Zakończenie obliczeń jest równoważne przejściu do stanu akceptującego lub odrzucającego
		\item Maszyna kończy każde swoje obliczenie
	\end{itemize}
	\begin{uwaga}
		Maszyna nie akceptuje wszystkich języków akceptowanych przez Maszynę Turinga w modelu 
		podstawowym, jet jej szczególnym przypadkiem
	\end{uwaga}
\end{df}

\begin{df}[Maszyna Turinga z wartownikiem]~\\
	Maszyną Turinga z wartownikiem nazywamy system $M = (Q, \Sigma, \Gamma, \delta, q_0, B, F)$
	z tym że alfabet taśmy posiada specjalny symbol $\#$ nazywany wartownikiem, zapisany w pierwszej 
	komórce taśmy.
	\begin{uwaga}
		Ta maszyna jest równoważna Maszynie Turinga w modelu podstawowym
	\end{uwaga}
\end{df}

\begin{df}[Maszyna Turinga z taśmą wielościeżkową]~\\
	Maszyną Turinga z taśmą wielościeżkową nazywamy system $M = (Q, \Sigma, \Gamma, \delta, q_0, B, F)$,
	gdzie funkcja przejścia zdefiniowana jest następująco
	\begin{equation}
		\delta : Q\times\Gamma^k\rightarrow G\times\Gamma\times\{L,P\}
	\end{equation}
\end{df}

\begin{tw}
	Maszyna Turinga z taśmą wielościeżkową jest równoważna Maszynie Turinga w modelu podstawowym (jednościeżkowej)
\begin{uwaga}
	Można ją zastąpić maszyną jednościeżkową kosztem zwiększenia alfabetu taśmy, który będzie
	iloczynem kartezjańskim alfabetów ścieżek.
	Aby formalnie udowodnić równoważność tych maszyn należałoby skonstruować maszynę symulującą
	działanie drugiej podobnie jak w Maszynie Turinga z wartownikiem.
\end{uwaga}
\end{tw}

\begin{tw}
	Maszyny Turinga z taśmą obustronnie nieograniczoną są równoważne Maszyną Turinga w modelu podstawowym
	
	\begin{proof}
	Wystarczy pokazać że dla maszyny jednego typu jesteśmy w stanie stworzyć maszynę drugiego typu akceptującą ten sam
	język.
	\begin{itemize}
		\item Maszyna z taśmą obustronnie nieograniczoną symulująca pracę maszyny w modelu podstawowym byłaby
		zdefiniowana identycznie jak symulowana maszyna.	W oczywisty sposób akceptowałaby ten sam język.
		\item Symulacja maszyny z taśmą obustronnie nieograniczoną przez maszynę w modelu podstawowym. 
		Ponieważ Maszyna Turinga wielościeżkowa jest równoważna maszynie w modelu podstawowym, pokażemy
		równoważność do wielościeżkowej. Zatem maszyna symulująca będzie maszyną dwuścieżkową, dodatkowo można pokusić się o 
		wprowadzenie wartownika. Koncept na realizację maszyny symulującej jest następujący:
			\begin{enumerate}
				\item Jeśli maszyna symulowana prowadzi obliczenia na lewo od pierwszego znaku lub wartownika
				to obliczenia prowadzone są na dolnej taśmie maszyny symulującej.
				\item W przeciwnym przypadku obliczenia prowadzone są na górnej taśmie.
				\item Należy wziąć pod uwagę inwersję ruchów w lewo/prawo na dolnej taśmie
			\end{enumerate}
	\end{itemize}
	\end{proof}
\end{tw}

\begin{df}[Wielotaśmowa Maszyna Turinga]~\\
	k-taśmową Maszyną Turinga nazywamy system
	\begin{equation}
		M = (Q, \Sigma, \Gamma_1\times\dots\times\Gamma_k, \delta, q_0, B, F)
	\end{equation}
	gdzie funkcja przejścia zdefiniowana jest następująco
	\begin{equation}
		\delta: Q\times(\Gamma_1\times\dots\times\Gamma_k) \rightarrow Q\times(\Gamma_1\times\dots\Gamma_k)\times\{L,R,S\}^k
	\end{equation}
\end{df}

\begin{tw}
	Klasa Maszyn Turinga wielotaśmowych jest równoważna klasie Maszyn Turinga w modelu podstawowym
	\begin{proof}
		Każda maszyna jednotaśmowa jest szczególnym przypadkiem maszyny wielotaśmowej. Ponieważ maszyna obustronnie
		ograniczona jest równoważna podstawowej to implikacja w jedną stronę jest trywialna.\\\\
		Pokażemy symylację wielotaśmowej maszyny yżywając Maszyny Turinga z taśmą obustronnie nieograniczoną o $2k$ 
		ścieżkach. Każdej taśmie będzie odpowiadać para ścieżek maszyny symulującej.
		Górna ścieżka każdej pary będzie zawierać symbol oznaczający położenie głowicy w maszynie symulowanej natomiast
		dolna będzie zawierać symulowaną taśmę.
		Symulacja będzie przebiegać następująco
		\begin{enumerate}
			\item Głowica maszyny symulującej znajduje się nad najbardziej na lewo wysuniętym symbolem -- znacznikiem głowic
			\item Głowica maszyny symulującej przesuwa się w prawo aż do najbardziej na prawo wysuniętego znacznika głowic.
			Zapamiętuje przy tym informację o symbolach czytanych przez głowicę maszyny wielotaśmowej -- 
			potrzebny jest odpowiednio duży zbiór stanów.
			\item Mając informację o stanie i symbolach pod głowicami maszyny wielotaśmowej, maszyna 
			symulująca zapamiętuje pełną informację o stanie Maszyny wielotaśmowej w odpowiadającym mu stanie.
			Oznacza to że maszyna symulująca jest teraz w stanie wyznaczyć opis chwilowy maszyny symulowanej 
			przed wykonaniem ruchu. Następnie stosując funkcje przejścia może wyznaczyć opis chwilowy po wykonaniu ruchu.
			\item Głowica maszyny symulującej przesuwa się do najbardziej na lewo wysuniętego symbolu oznaczającego położenie
			głowicy w modelu wielotaśmowym (znacznik głowicy). W trakcie tego powrotu maszyna symulująca uaktualnia odpowiednie
			komórki ścieżek opisujące taśmy maszyny symulowanej, a także położenie znaczników głowic.
			\item Głowica maszyny symulującej zatrzyma się na najbardziej na lewo położonym znacznikiem głowic 
			oraz zmieni stan zgodnie ze mianą stanu maszyny symulowanej.
		\end{enumerate}
	\end{proof}
	\begin{uwaga}
		Symulacja maszyny wielotaśmowej jednotaśmową skutkuje wzrostem złożoności obliczeniowej z kwadratem ruchów.
	\end{uwaga}
\end{tw}

\begin{df}[Niedeterministyczna Maszyna Turinga]~\\
	Definicja jest analogiczna do Maszyny Turinga w modelu podstawowym z tym że funkcja przejścia
	zdefiniowana jest następująco
	\begin{equation}
		\delta: Q\times\Gamma \rightarrow \bigcup^\infty_{k=0} (Q\times\Gamma\times\{L,R\})^k\quad
		\footnote{Dla $k=0$ wartość jest nieokreślona}
	\end{equation}
	Zatem ruch tej maszyny polega na niedeterministycznym wyborze jednej z wartości funkcji przejścia,
	a następnie wykonaniu ruchu takiego jaki wykonałaby deterministyczna maszyna.
	\begin{uwaga}
		Obliczenie niedeterministycznej Maszyny Turinga jest drzewem w którym
		\begin{itemize}
			\item wierzchołki etykietowane są opisem chwilowym maszyny	
			\item potomkami dowolnego wierzchołka są wierzchołki pozostające z nim w relacji ruchu
		\end{itemize}
	\end{uwaga}
\end{df}

	
\end{document}
