\documentclass[12pt,a4paper]{article}

\usepackage[utf8]{inputenc}
\usepackage[T1]{fontenc}
\usepackage[a4paper]{geometry}

\usepackage{pdflscape}
\usepackage{float}

\usepackage{color}

\usepackage{amsthm}
\usepackage{amsfonts}
\usepackage{amsmath} %pakiet potrzebny do \eqref
\usepackage{amssymb} %pakiet potrzebny do \blacksquare i \mathbb{•}

\usepackage[polish]{babel}

\usepackage{graphicx}

\usepackage[normalem]{ulem}

\usepackage{listliketab}
\usepackage{paralist}

\newtheorem{lemat}{Lemat}
\newtheorem{tw}{Twierdzenie}
\newtheorem{przyklad}{Przykład}
\newtheorem{zad}{Zadanie}
\theoremstyle{definition}
\newtheorem{df}{Definicja}
\theoremstyle{remark}
\newtheorem{uwaga}{Uwaga}

\usepackage[pdfauthor={Janisz},%
pdftitle={Teoria Automatów i Języków},%
pagebackref=true,%
pdftex]{hyperref}
\hypersetup{colorlinks=false}


\pagestyle{plain}
\begin{document}
\title{ Teoria Automatów i Języków}
\author{\vspace{-5ex}}
\date{\today}
\maketitle
\tableofcontents

\DeclareGraphicsExtensions{.pdf,.png,.jpg}
\begin{center}
\leavevmode

\vfill

\includegraphics[width=1 in]{by-sa.png}
\end{center}
\label{fig:cc}
\scriptsize{Ten utwór jest dostępny na licencji  
\href{http://creativecommons.org/licenses/by-sa/3.0/pl/}{Creative Commons Uznanie autorstwa-Na tych samych warunkach 3.0 Polska.}}

\pagebreak

\section{Relacje}

\begin{df}
Niech $X_1, X_2, \dots, X_n$ będą zbiorami. Podzbiór \href{http://pl.wikipedia.org/wiki/Iloczyn_kartezja%C5%84ski}{iloczynu kartezjańskiego} 
$R \subset X_1 \times X_2 \times \dots \times X_n$ 
nazywamy n-argumentową \href{http://pl.wikipedia.org/wiki/Relacja_(matematyka)}{relacją}. 
\end{df}

\begin{df}
Niech $X, Y$ będą zbiorami. Wtedy \href{http://pl.wikipedia.org/wiki/Relacja_dwuargumentowa}{relację dwuargumentową} nazywamy $\rho = X \times Y$, a zbiory $X, Y$ odpowiednio dziedziną i przeciwdziedziną.  
\end{df}

\begin{df}
Relacją binarną nazywamy taką relację dwuargumentową,
w której dziedzina i przeciwdziedzina są równe. $\rho = X \times X$
\end{df}

\subsection{Własności relacji}

$\forall x,y,z \in X$
\begin{listliketab}
\storestyleof{itemize} 
\begin{tabular}{ll}
	\textbullet  zwrotna & $x \rho x$ \\
	\textbullet  przeciw-zwrotna & $\neg x\rho x$ \\
	\textbullet  symetryczna & $x\rho y \Rightarrow y\rho x$ \\
	\textbullet  przeciw-symetryczna & $x \rho y \Rightarrow \neg y\rho x $ \\
	\textbullet  antysymetryczna & $x\rho y \wedge y\rho x \Rightarrow x= z$ \\
	\textbullet  przechodnia & $x\rho y \wedge y\rho z \Rightarrow x\rho z$ \\
	\textbullet  spójna & $ x\rho y \vee y\rho x $ \\
\end{tabular}
\end{listliketab}

\begin{df}
Relację nazywamy \href{http://pl.wikipedia.org/wiki/Relacja_r%C3%B3wnowa%C5%BCno%C5%9Bci}{relacją równoważności}
jeśli jest jednocześnie zwrotna, symetryczna i przechodnia
\end{df}

\begin{df}
Niech $\rho\subset X\times X$ będzie relacją binarną, a $\mathcal{R}$ zbiorem własności relacji. Powiemy że $\rho'$ jest domknięciem relacji $\rho$
ze względu na $\mathcal{R} \quad \Leftrightarrow$
\begin{enumerate}
	\item $\rho \subset \rho'$
	\item $\rho'$ jest domknięta ze względu na własności z $\mathcal{R}$
	\item $\rho'$ jest najmniejszą relacją spełniającą powyższe warunki
\end{enumerate}
\end{df}

\begin{przyklad}
	$\rho \subset \mathbb{N}\times\mathbb{N} \quad m,n \in \mathbb{N} \quad m\rho n \equiv m +1 = n$ \\
	\begin{center}
	\begin{tabular}{ccccccccc}
		$\rho$ & 0 & 1 & 2 & 3 & 4 & 5 & 6 & \dots \\
			0  & {\color{red}1} & {\color{blue}1} & {\color{green}1}  & {\color{green}1}  & {\color{green}1}  & {\color{green}1}  & {\color{green}1}  & \dots \\
			1  & {\color{yellow}0}  & {\color{red}1}  & {\color{blue}1} & {\color{green}1}  & {\color{green}1}  & {\color{green}1}  & {\color{green}1}  & \dots \\
			2  & {\color{yellow}0}  & {\color{yellow}0}  & {\color{red}1}  & {\color{blue}1} & {\color{green}1}  &  {\color{green}1} & {\color{green}1}  & \dots \\
			3  & {\color{yellow}0}  & {\color{yellow}0}  & {\color{yellow}0}  & {\color{red}1}  & {\color{blue}1} & {\color{green}1}  & {\color{green}1}  & \dots \\
			4  & {\color{yellow}0}  & {\color{yellow}0}  & {\color{yellow}0}  & {\color{yellow}0}  & {\color{red}1}  & {\color{blue}1} & {\color{green}1}  & \dots \\
			5  & {\color{yellow}0}  & {\color{yellow}0}  & {\color{yellow}0}  & {\color{yellow}0}  & {\color{yellow}0}  & {\color{red}1}  & {\color{blue}1} & \dots \\
	     \dots & \dots  & \dots  & \dots  & \dots  & \dots  & \dots  & {\color{red}1}  & \dots \\
	\end{tabular}\\
	\end{center}	
	{\color{yellow} $0$ oznacza brak relacji}, $1$ oznacza że dwa elementy są w relacji {\color{red}zwrotnej}, {\color{green}przechodniej} lub {\color{blue}$\rho$}. 
	Cała tabel przedstawia natomiast relację $\rho'$ taką że $\forall m,n \in \mathbb{N} \quad m \rho' n \equiv m \leqslant n$,  będącą domknięciem relacji $\rho$ ze względu na 
	$\mathcal{R} = \{\text{zwrotność, przechodniość}\}$
\end{przyklad}

\subsection{Słowa i alfabety}
	\begin{df}
		Dowolny skończony ciąg nad danym alfabetem nazywamy \href{http://pl.wikipedia.org/wiki/S%C5%82owo_(matematyka)}{słowem}.
		 $\varepsilon$ -- słowo puste, $\Sigma^\star$ -- zbiór wszystkich słów
	\end{df}
	
	\begin{df}
		Dowolny podzbiór zbioru słów jest językiem $L \subset \Sigma^\star$
	\end{df}

	\begin{przyklad}~\\
		\begin{itemize}
			\item $\Sigma = \{0, 1, 2, 3, 4, 5, 6, 7, 8, 9\}$
			\item $\Sigma^\star = \{\varepsilon, 0, 1, \dots, 9, 00, 01, \dots, 09, 10, 11, \dots, 99, 100, \dots\}$
			\item $L = \{0, 1, 2, \dots, 9, 10, \dots, 19, 20, 21, \dots, 99, 100, \dots \}$
			\item $L' = \{2, 3, 5, 7, 11, \dots \}$
		\end{itemize}
	\end{przyklad}

	\begin{df}
		Jeżeli nad zbiorem $X$ zdefiniowano relację równoważności $\sim$ to klasą abstrakcji elementu $x\in X$ nazwiemy zbiór wszystkich elementów 
		z $X$ które są w relacji z $x$: $[x]_\sim = \{y\in X: y\sim x\}$
	\end{df}		
	
	\begin{przyklad}
		$\Sigma = \{0, 1, \dots, 9\}$, $\rho \subset \Sigma^\star \times \Sigma^\star \quad \forall n,m\in \Sigma^\star \quad m\rho n \equiv
		 \text{wartość } m = \text{wartość } n$. Klasy abstrakcji: \\
		\begin{itemize}
			\item $A_\epsilon = \{\epsilon\}$
			\item $A_0 = \{0, 00, 000, \dots\}$
			\item $A_1 = \{1, 01, 001, \dots\}$
			\item $A_2 = \{2, 02, 002, \dots\}$
		\end{itemize}
	\end{przyklad}		
	
	\begin{przyklad}
		$\rho \subset \mathbb{N} \times \mathbb{N}:  m\rho n \equiv |m-n| = 3$ -- nie jest zwrotna ani przechodnia ale jest symetryczna.
		Domknięcie zwrotne: $m\rho n \equiv (m-n) mod 3 = 0 \vee m = \epsilon = n$. Klasy tego domknięcia abstrakcji $[\epsilon]$, $[0]$, $[1]$, $[2]$
	\end{przyklad}
	
	\begin{df}
		Powiemy że relacja $\rho$ jest prawostronnie niezmienna $\Leftrightarrow$ gdy dla dowolnych dwóch słów będących w relacji, po dopisaniu do obu tego samego słowa ze zbiru słów nadal pozostaną w relacji. 
		$$ (\forall u, v \in \Sigma^\star) \quad[u \rho v \Rightarrow (\forall z\in\Sigma^\star) uz \rho vz] $$
	\end{df}
	
\subsection{Relacje indukowane przez język}	
	
	\begin{df}
		Powiemy że $R_L$ jest relacją indukowaną przez język 
		$L \Leftrightarrow$ $$ (\forall u,v \in \Sigma^\star) \quad \{uR_Lv \equiv [(\forall z\in \Sigma^\star) \quad uz\in L \equiv vz\in L]\}$$
	\end{df}	
	
	\begin{zad} 
		Udowodnij, że relacja indukowana przez język $(R_L)$ jest relacją równoważności.
	\end{zad}
	
%Wykład 2 
%12.10.2012

	\begin{przyklad}
		Niech $L \subset \Sigma^\star$ -- język, $R_L \subset \Sigma^\star \times \Sigma^\star$\footnote{relacja jest nad zbiorem wszystkich słów z alfabetu,
		a nie nad językiem dlatego przy wyznaczaniu klas abstrakcji należy zbadać również elementy nie należące do języka}
		$R_L$ jest relacją określoną przez język $L$ w następując sposób $$L \equiv (\forall u,v \in \Sigma^\star) \quad uR_Lv \equiv [(\forall z\in \Sigma^\star) ~
		uz \in L \equiv vz \in L ]  $$
		Niech alfabet będzie alfabetem binarnym $(\Sigma = (0, 1)$, a język $L$ językiem binarnym bez znaczących zer -- $L = \{0, 1, 10, 11, 100, \dots \}$
		\footnote{słowo puste nie należy do języka}. Wtedy relacja $R_L$ tworzy następujące klasy abstrakcji:
		\begin{enumerate}
			\item $A_\epsilon = \{\epsilon\}$ -- klasa zawierająca tylko słowo puste $(\epsilon)$ 
			\item $A_0 = \{0\}$ -- klasa zawierająca tylko $0$
				\begin{itemize}
					\item $0$ jest w relacji z samym sobą
					\item nie jest w relacji z żadnym innym słowem z poza języka ponieważ:
					\begin{proof}
						Niech $z = \epsilon$ wówczas $0z = 0\epsilon = 0 \in L$ oraz $uz = u\epsilon = u \not\in L$. Zatem nie może być w relacji z żadnym 
						słowem z poza języka
					\end{proof}
					\item nie jest w relacji z żadnym słowem z języka bo:
					\begin{proof}
						Niech $z = 1$ wówczas $0z = 01 \not\in L$ oraz $uz = u1 = 1\dots 1 \in L$
					\end{proof}
				\end{itemize}
			\item $A_{10} = L - \{0\}$ -- klasa zawierająca wszystkie słowa z języka poza $0$
				\begin{itemize}
					\item Każdy element jest w relacji z elementem z klasy
					\begin{proof}
						Niech $u, v \in A_{10}$ wówczas $u = 1\dots$ i $v = 1\dots$. Dla $z \in \Sigma^\star \quad uz = 1\dots$ i $vz = 1\dots$
						stąd $uz \in L$ i $vz \in L$
					\end{proof}
					\item Każdy element nie jest w relacji z elementem nie należącym do $A_{10}$
					\begin{proof}
						Niech $u \in A_{10}$ i $v \not\in A_{10}$ wówczas $u = 1\dots$ 
						\begin{itemize}
							\item jeśli $v\ \neq \epsilon$ to znaczy $v = 0\dots $, wówczas dla $z = 1 \quad uz \in L \wedge vz \not\in L$
							\item jeśli $v = \epsilon$ to dla $z = \epsilon \quad uz \in L \wedge vz \not\in L$
							\footnote{Można skorzystać z tego że już udowodniliśmy że $\epsilon$ jest w innej klasie abstrakcji}
						\end{itemize}
					\end{proof}
				\end{itemize}
			\item $A_{01} = \Sigma^\star - (L \cup {\epsilon}) = \{ \text{słowa z wiodącymi nieznaczącymi zerami} \}$
		\end{enumerate}
		W przypadku alfabetu złożonego z większej liczby znaków dla tej relacji $R_L$, klasy abstrakcji byłyby identyczne.
	\end{przyklad}	
	
	\begin{przyklad}
		$L = $ zbiór słów takich że kolejne trójki liczb składają się z identycznych liter $ = \{\epsilon, 000, 111, 000111, 000000, \dots\}\\
		L \subset \{0, 1\}^\star$\\
		Klasy abstrakcji relacji indukowanej przez język $L$
		\begin{enumerate}
			\item $A_L = \{L\}$ -- wszystkie słowa z języka
			\item $A_0 = \{u0: ~ u\in L\}$ -- słowa z języka z dodatkowym $0$ na końcu
			\item $A_1 = \{u1: ~ u\in L\}$ -- słowa z języka z dodatkowym $1$ na końcu
			\item $A_{00} = \{u00: ~ u\in L\}$ -- słowa z języka z dodatkowym $00$ na końcu
			\item $A_{11} = \{u11: ~ u\in L\}$ -- słowa z języka z dodatkowym $11$ na końcu
			\item $A_\sim$ -- wszystkie pozostałe słowa
		\end{enumerate}
	\end{przyklad}		
	
	\begin{przyklad}
		$L = $ zbiór słów które mają tyle samo zer i jedynek $ = \{\epsilon, 01, 10, 0011, 1010, \dots\}\\
		L \subset \{0, 1\}^\star$\\
		Klasy abstrakcji relacji indukowanej przez język $L$\footnote{indeks dolny przy klasie abstrakcji to różnica pomiędzy liczą zer i jedynek}
		\begin{enumerate}
			\item $A_0 = L$
			\item $A_1 = \{0, \dots \}$
			\item $A_-1 = \{1, \dots \}$
			\item $A_2 = \{11, \dots \}$\\
			\vdots
		\end{enumerate}
	\end{przyklad}
	\begin{zad}
		Podaj klasy abstrakcji dla relacji indukowanej przez język palindromów nad alfabetem binarnym
	\end{zad}

\section{Grafy}
	
	\begin{df}
		\href{http://pl.wikipedia.org/wiki/Graf_(matematyka)}{Grafem} nazywamy parę $G = (V, E)$ gdzie $V$ to zbiór wierzchołków 
		$E \subset V \times V$ to zbiór krawędzi.
	\end{df}	

	\begin{df}
		\href{http://pl.wikipedia.org/wiki/Drzewo_(matematyka)}{Drzewo} to graf spójny graf acykliczny o następujących własnościach:
		\begin{enumerate}
			\item Jednowierzchołkowy graf $G = (\{v\}, \emptyset)$ nazywamy korzeniem drzewa
			\item Jeśli $T_1 = (v_1, E_1), ~ T_2 = (v_2, E_2), ~\dots ,~ T_k = (v_k, E_k)$ są drzewami o korzeniach $v_{01}, v_{02}, \dots, v_{0k}$
			to $$T = \underbrace{ \left(\{v_0\} \cup \bigcup^k_{i=1}V_i \right. }_\text{wierzchołki} ~,~
					\underbrace{ \left. \bigcup_{i=0}^k \{v_0, v_{0i}\} \cup \bigcup_{i=0}^k E_i \right) }_\text{krawędzie}$$
			\item Dowolną konstrukcję otrzymaną przez zastosowanie reguł 1 i 2
		\end{enumerate}			
	\end{df}
	
	\begin{df}
		Jeśli wysokościami drzew $T_1 = (V_1, E_1), ~ T_2 = (V_2, E_2), ~\dots ,~ T_k = (V_k, E_k)$ są $h_1, h_2, \dots , h_k$ to
		wysokość drzewa $T$ wynosi $h = max\{h_1, h_2, \dots , h_k\}$
	\end{df}
	
	\begin{df}
		K-drzewo to drzewo którego dowolny wierzchołek ma k następników.
	\end{df}
	
	\begin{df}[Zasada indukcji matematycznej]
		Niech $W$ będzie pewną własnością liczb naturalnych taką że:
		\begin{enumerate}
			\item $W(0)$ -- własność $W$ zachodzi dla $0$
			\item $(\forall k=0, 1, \dots) \quad W(k) \Rightarrow W(k+1)$ 
		\end{enumerate}
		Wówczas $\forall n \in \mathbb{N} \quad W(n)$ --własność $W$ zachodzi dla n
	\end{df}
	
	\begin{lemat}
		Dowolne k-drzewo o wysokości $h$ ma nie więcej niż $k^h$ liści
		\begin{proof} (\emph{indukcyjny})\\
			\begin{enumerate}
				\item dla drzewa jednowierzchołkowego liczba liści wynosi $1$, a wysokość $0$
				\item niech $T_1, T_2, \dots, T_l$ -- k-drzewa o wysokościach $h_1, h_2, \dots, h_l$
				Niech $T$ -- drzewo zbudowane z $T_1, T_2, \dots, T_l$ według reguły z definicji drzewa. 
				Wysokość $T$ jest równa $1 + max\{h_1, h_2, \dots, h_l\}$.\\
				Zakładam, że liczby liści w drzewach $T_1, \dots, T_l$ są nie większe niż $k^{h_1}, k^{h_2}, \dots, k^{h_l}$. 
				Liczba liści w drzewie $T$ jest nie większa niż 
				$$k^{h_1} + k^{h_2} + \dots + k^{h_l} \leqslant k\cdot max\{k^{h_1}, k^{h_2}, \dots, k^{h_l}\} = k\cdot k^{max\{h_1, h_2, \dots, h_l\}} 
				= k^{1 + max\{h_1, h_2, \dots, h_l\}} = k^h$$
				\item Na mocy spełnienia punktu 1 i 2 zasady indukcji matematycznej wnioskujemy iż lemat jest prawdziwy.
			\end{enumerate}
		\end{proof}
	\end{lemat}
	
\section{Wyrażenia regularne}
	
	\begin{df}(\href{http://pl.wikipedia.org/wiki/Domkni%C4%99cie_Kleene'ego}{Domknięcie Kleene'ego})
	Jeżeli $L$ jest językiem nad pewnym alfabetem to:
		\begin{itemize}
			\item $L^0 = \{\epsilon\}$
			\item $L^{n+1} = L^n \circ L$
			\item $L^* = \bigcup\limits_{k=0}^\infty L^k$
		\end{itemize}
	\end{df}	
	
	\begin{przyklad}
		Niech $L = \{00, 000\}$ będzie alfabetem, wtedy:\\
		\begin{inparaenum}[\itshape a\upshape)] 
			\item $L^0 = \{\epsilon\}$
			\item $L^1 = \{00, 000\}$
			\item $L^2 = \{0000, 00000, 000000\}$
			\item $L^* = \{\epsilon, 00, 000, 0000, 00000, \dots\}$
		\end{inparaenum}
	\end{przyklad}
	
	\begin{df} \href{http://pl.wikipedia.org/wiki/Wyra%C5%BCenia_regularne}{Wyrażeniem regularnym} nad alfabetem $\Sigma$ nazywamy konstrukcje
	 powstałe w wyniku skończonej iteracji poniższych zasad.
		\begin{enumerate}
			\item Podstawowe wyrażenia regularne: \footnote{Poniższe znaki powinny zostać wyróżnione, aby zaznaczyć używanie ich w kontekście
			wyrażeń regularnych. Jednak ponieważ zazwyczaj ich znaczenie wynika wprost z kontekstu, nie zastosowano tu żadnego wyróżnienia}
				\begin{itemize}
					\item $\emptyset$ -- brak znaku
					\item $\epsilon$ -- słowo puste
					\item $\Sigma$ -- alfabet $\forall a\in\Sigma ~~ a$
				\end{itemize}
			\item Jeśli $r$ i $s$ są wyrażeniami regularnymi to następujące wyrażenia też są wyrażeniami regularnymi
				\footnote{W dalszej części przyjmujemy natępujące priorytety operacji (od najniższego) suma, konkatenacja, domknięcie Kleene'ego. 
				$((a+b)(a+(b^*))) \equiv (a+b)(c+b^*)$}
				\begin{itemize}
					\item $(r+s)$ -- suma
					\item $(rs)$ -- konkatenacja
					\item $(r^*)$ -- domknięcie Kleene'ego
				\end{itemize}
		\end{enumerate}		
	\end{df}	
	
	\begin{uwaga}
	Wyrażenia regularne generują języki.
	\end{uwaga}
	
	\begin{tw}
		Jeżeli $r$ i $s$ genrują języki $R$ oraz $S$ to:
		\begin{itemize}
			\item $r+s$ generuje $R\cup S$
			\item $rs$ generuje $RS = \{uv: ~ u\in R, ~ v\in S\}$
			\item $r^*$ generuje $R^*$
		\end{itemize}
	\end{tw}
	
	\begin{df}
		Językami regularnymi nazywamy takie i tylko takie języki, które zostały wygenerowane przez wyrażenie regularne
	\end{df}	
	
	\begin{przyklad}
		Sprawdzić czy Język liczb dziesiętnych bez nieznaczących zer jest językiem regularnym. \\
		\emph{Odp.} Tak, generuje go następujące wyrażenie regularne:
		$$  0 + (1+2+\dots+9)(0+1+2+\dots+9)^* $$
		\begin{proof}
			Należy wykazać że każde słowo generowane przez to wyrażenie jest liczbą dziesiętną, bez nieznaczących zer oraz każda liczba 
			dziesiętna bez nieznaczących zer jest generowana przez to wyrażenie.\\
			Generujemy $0$ lub liczbę rozpoczynającą się do cyfry $> 0$, po której następuje dowolna ilość dowolnych cyfr.
		\end{proof}
	\end{przyklad}		
	
	\begin{przyklad}
		Wykazać że język słów binarnych w których sekwencja trzech kolejnych jedynek występuje co najmniej raz,
		jest językiem regularnym. \\
		\emph{Odp.} Tak, generuje go następujące wyrażenie regularne:
		$$[00^*](\epsilon+1+11)00]111[00^*(\epsilon+1+11)]^* $$
	\end{przyklad}
	
	\begin{zad}
		Sprawdzić czy język słów binarnych, które zaczynają i kończą się tą samą literą jest językiem regularnym.
	\end{zad}
	\begin{zad}
		Sprawdzić czy język słów binarnych, którego kolejne sekwencje tych samych liter są nie krótsze niż poprzednie jest językiem regularnym.
	\end{zad}
	
	\subsection{Tożsamości wyrażeń regularnych}
	\begin{eqnarray}
		\emptyset + r &=& r + \emptyset \\
		r + s &=& s + r \\
		(r+s) + t &=& r + (s+t) \\
		\epsilon + r &=& r \\
		(rs)t &=& r(st) \\
		r(s+t) &=& rs + rt \\
		(r^*)^* &=& r^* \\
		(r^*s^*)^* &=& (r+s)^* \\
		(\epsilon + r)^* &=& r^* \\
		(r^* + s^*)^* &=& (r + s)^*\\		
	\end{eqnarray}
	
	\begin{zad}
		Udowodnić powyższe tożsamości
	\end{zad}
	
	\begin{lemat}{\href{http://pl.wikipedia.org/wiki/Twierdzenie_Myhilla-Nerode%27a}{Myhill-Nerode}}
		Język $L$ jest regularny $\Leftrightarrow$ liczba kals abstrakcji relacji $R_L$ indukowanej przez ten język jest skończona.
	\end{lemat}	
	
	\begin{lemat}{\href{http://pl.wikipedia.org/wiki/Lemat_o_pompowaniu_dla_j%C4%99zyk%C3%B3w_regularnych}{O pompowaniu}}
		Jeśli język $L$ jest regularny to istnieje stała $n_L$ taka że dla każdego słowa $z\in L$ spełniona jest implikacja
		\footnote{jeśli słowo jest dłuższe niż $n_L$ to istnieje taki podział słowa $z$ na 3 części taki, że suma długości pierwszej
		i drugiej jest mniejsza niż $n_i$ i trzecia nie jest pusta. Wtedy słowo $z$ złożone z częśi pierwszej połączonej z powieloną 
		częścią drugą i zakończone częścią trzecią należy do $L$}
		$$|z| \geqslant n_L \Rightarrow (\exists u,v,w ~~ z = uvw ~~ |uv| \leqslant n_L, ~~ |v| \geqslant 1)
		(\forall i\in\mathbb{N}) z_i = uv^iw \in L$$
	\end{lemat}
	
	\begin{lemat}{Kontrapozycja lematu o pompowaniu}
		Jeśli dla dowolnej stałej $n$ istnieje słowo $z\in L$ takie że 
		$|z|\geqslant n$ i $(\forall u,v,z: ~~ z = uvw ~~ |uv| < n, ~ |v| \geqslant 1)(\exists i\in\mathbb{N}) ~~ z = uv^iw \not\in L$
		to język $L$ nie jest regularny.
	\end{lemat}


	\begin{przyklad}
		Sprawdzić żę język palindromów nad alfabetem binarnym nie jest językiem regularnym.
		\begin{proof}
			\begin{itemize}
				\item[sposób 1] Wystarczy wskazać nieskończony ciąg klas abstrakcji aby udowodnić że język nie jest regularny.
				Pokażę że nieskończony ciąg słów z których każde jest w innej klasie abstrakcji: 
				$z_0 = \epsilon, ~ z_1 = 0, ~ z_2 = 00, ~\dots$.\\
				Dowolne 2 słowa tego ciągu należą do różnych klas abstrakcji.
				$z_i = 0^i \quad z_j = 0^j ~~ z \neq j, ~~ z_i  \xout{R_L}z_j$ bo dla $w = 10^i ~~
				\begin{matrix}
					z_iw \in L \\ z_jw \not\in L
				\end{matrix}$
				\item[sposób 2] Niech $N$ -- stała, z kontrapozycji lematu o pompowaniu. Niech $z = 0^N10^N$, dowolny podział
				spełniający założenia tego lematu to taki którego środkowa część podziału będzie się składać
				z zer i początkowej części słowa.
				$$z = \underbrace{0^p}_u\underbrace{0^q}_v\underbrace{0^r10^N}_w\text{ , gdy }
				\left\lbrace \begin{matrix}
					p+q+r &=& N \\ p + q &\leqslant& N \\ 1 &\leqslant& q
				\end{matrix} \right.$$
				 Niech $i=0$ wówczas $z_0 = 0^p0^r10^N ~~ p+r < N \Rightarrow z_0 \not\in L$
			\end{itemize}
		\end{proof}
	\end{przyklad}	
	
	\begin{zad}
		Sprawdzić czy język słów binarnych $L = \{ w\in \{0,1\}^*: ~ \#_0w < \#_1w\}$\footnote{$\#$ oznacza liczbę wystąpień znaku z 
		indeksu dolnego w słowie}, jest regularny.
	\end{zad}
	\begin{zad}
		Sprawdzić czy język słów binarnych które składają się z sekwencji tych samych liter nie krótszych niż trzy, jest regularny.
	\end{zad}
	
\end{document}
