\section{Wyrażenia regularne}
	
	\begin{df}(\href{http://pl.wikipedia.org/wiki/Domkni%C4%99cie_Kleene'ego}{Domknięcie Kleene'ego})
	Jeżeli $L$ jest językiem nad pewnym alfabetem to:
		\begin{itemize}
			\item $L^0 = \{\epsilon\}$
			\item $L^{n+1} = L^n \circ L$
			\item $L^* = \bigcup\limits_{k=0}^\infty L^k$
		\end{itemize}
	\end{df}	
	
	\begin{przyklad}
		Niech $L = \{00, 000\}$ będzie alfabetem, wtedy:\\
		\begin{inparaenum}[\itshape a\upshape)] 
			\item $L^0 = \{\epsilon\}$
			\item $L^1 = \{00, 000\}$
			\item $L^2 = \{0000, 00000, 000000\}$
			\item $L^* = \{\epsilon, 00, 000, 0000, 00000, \dots\}$
		\end{inparaenum}
	\end{przyklad}
	
	\begin{df} \href{http://pl.wikipedia.org/wiki/Wyra%C5%BCenia_regularne}{Wyrażeniem regularnym} nad alfabetem $\Sigma$ nazywamy konstrukcje
	 powstałe w wyniku skończonej iteracji poniższych zasad.
		\begin{enumerate}
			\item Podstawowe wyrażenia regularne: \footnote{Poniższe znaki powinny zostać wyróżnione, aby zaznaczyć używanie ich w kontekście
			wyrażeń regularnych. Jednak ponieważ zazwyczaj ich znaczenie wynika wprost z kontekstu, nie zastosowano tu żadnego wyróżnienia}
				\begin{itemize}
					\item $\emptyset$ -- brak znaku
					\item $\epsilon$ -- słowo puste
					\item $\Sigma$ -- alfabet $\forall a\in\Sigma ~~ a$
				\end{itemize}
			\item Jeśli $r$ i $s$ są wyrażeniami regularnymi to następujące wyrażenia też są wyrażeniami regularnymi
				\footnote{W dalszej części przyjmujemy natępujące priorytety operacji (od najniższego) suma, konkatenacja, domknięcie Kleene'ego. 
				$((a+b)(a+(b^*))) \equiv (a+b)(c+b^*)$}
				\begin{itemize}
					\item $(r+s)$ -- suma
					\item $(rs)$ -- konkatenacja
					\item $(r^*)$ -- domknięcie Kleene'ego
				\end{itemize}
		\end{enumerate}		
	\end{df}	
	
	\begin{uwaga}
	Wyrażenia regularne generują języki.
	\end{uwaga}
	
	\begin{tw}
		Jeżeli $r$ i $s$ genrują języki $R$ oraz $S$ to:
		\begin{itemize}
			\item $r+s$ generuje $R\cup S$
			\item $rs$ generuje $RS = \{uv: ~ u\in R, ~ v\in S\}$
			\item $r^*$ generuje $R^*$
		\end{itemize}
	\end{tw}
	
	\begin{df}
		Językami regularnymi nazywamy takie i tylko takie języki, które zostały wygenerowane przez wyrażenie regularne
	\end{df}	
	
	\begin{przyklad}
		Sprawdzić czy Język liczb dziesiętnych bez nieznaczących zer jest językiem regularnym. \\
		\emph{Odp.} Tak, generuje go następujące wyrażenie regularne:
		$$  0 + (1+2+\dots+9)(0+1+2+\dots+9)^* $$
		\begin{proof}
			Należy wykazać że każde słowo generowane przez to wyrażenie jest liczbą dziesiętną, bez nieznaczących zer oraz każda liczba 
			dziesiętna bez nieznaczących zer jest generowana przez to wyrażenie.\\
			Generujemy $0$ lub liczbę rozpoczynającą się do cyfry $> 0$, po której następuje dowolna ilość dowolnych cyfr.
		\end{proof}
	\end{przyklad}		
	
	\begin{przyklad}
		Wykazać że język słów binarnych w których sekwencja trzech kolejnych jedynek występuje co najmniej raz,
		jest językiem regularnym. \\
		\emph{Odp.} Tak, generuje go następujące wyrażenie regularne:
		$$[00^*](\epsilon+1+11)00]111[00^*(\epsilon+1+11)]^* $$
	\end{przyklad}
	
	\begin{zad}
		Sprawdzić czy język słów binarnych, które zaczynają i kończą się tą samą literą jest językiem regularnym.
	\end{zad}
	\begin{zad}
		Sprawdzić czy język słów binarnych, którego kolejne sekwencje tych samych liter są nie krótsze niż poprzednie jest językiem regularnym.
	\end{zad}
	
	\subsection{Tożsamości wyrażeń regularnych}
	\begin{eqnarray}
		\emptyset + r &=& r + \emptyset \\
		r + s &=& s + r \\
		(r+s) + t &=& r + (s+t) \\
		\epsilon + r &=& r \\
		(rs)t &=& r(st) \\
		r(s+t) &=& rs + rt \\
		(r^*)^* &=& r^* \\
		(r^*s^*)^* &=& (r+s)^* \\
		(\epsilon + r)^* &=& r^* \\
		(r^* + s^*)^* &=& (r + s)^*\\		
	\end{eqnarray}
	
	\begin{zad}
		Udowodnić powyższe tożsamości
	\end{zad}
	
	\begin{lemat}{\href{http://pl.wikipedia.org/wiki/Twierdzenie_Myhilla-Nerode%27a}{Myhill-Nerode}}
		Język $L$ jest regularny $\Leftrightarrow$ liczba kals abstrakcji relacji $R_L$ indukowanej przez ten język jest skończona.
	\end{lemat}	
	
	\begin{lemat}{\href{http://pl.wikipedia.org/wiki/Lemat_o_pompowaniu_dla_j\%C4\%99zyk\%C3\%B3w_regularnych}{O pompowaniu}}
		Jeśli język $L$ jest regularny to istnieje stała $n_L$ taka że dla każdego słowa $z\in L$ spełniona jest implikacja
		\footnote{jeśli słowo jest dłuższe niż $n_L$ to istnieje taki podział słowa $z$ na 3 części taki, że suma długości pierwszej
		i drugiej jest mniejsza niż $n_i$ i trzecia nie jest pusta. Wtedy słowo $z$ złożone z częśi pierwszej połączonej z powieloną 
		częścią drugą i zakończone częścią trzecią należy do $L$}
		\begin{eqnarray}
			|z| \geqslant n_L \Rightarrow (\exists u,v,w ~~ z = uvw ~~ |uv| \leqslant n_L, ~~ |v| \geqslant 1)
			(\forall i\in\mathbb{N}) z_i = uv^iw \in L
		\end{eqnarray}
	\end{lemat}
	
	\begin{lemat}{Kontrapozycja lematu o pompowaniu}
		Jeśli dla dowolnej stałej $n$ istnieje słowo $z\in L$ takie że 
		$|z|\geqslant n$ i $(\forall u,v,z: ~~ z = uvw ~~ |uv| < n, ~ |v| \geqslant 1)(\exists i\in\mathbb{N}) ~~ z = uv^iw \not\in L$
		to język $L$ nie jest regularny.
	\end{lemat}


	\begin{przyklad}
		Sprawdzić żę język palindromów nad alfabetem binarnym nie jest językiem regularnym.
		\begin{proof}
			\begin{itemize}
				\item[sposób 1] Wystarczy wskazać nieskończony ciąg klas abstrakcji aby udowodnić że język nie jest regularny.
				Pokażę że nieskończony ciąg słów z których każde jest w innej klasie abstrakcji: 
				$z_0 = \epsilon, ~ z_1 = 0, ~ z_2 = 00, ~\dots$.\\
				Dowolne 2 słowa tego ciągu należą do różnych klas abstrakcji.
				$z_i = 0^i \quad z_j = 0^j ~~ z \neq j, ~~ z_i  \xout{R_L}z_j$ bo dla $w = 10^i ~~
				\begin{matrix}
					z_iw \in L \\ z_jw \not\in L
				\end{matrix}$
				\item[sposób 2] Niech $N$ -- stała, z kontrapozycji lematu o pompowaniu. Niech $z = 0^N10^N$, dowolny podział
				spełniający założenia tego lematu to taki którego środkowa część podziału będzie się składać
				z zer i początkowej części słowa.
				$$z = \underbrace{0^p}_u\underbrace{0^q}_v\underbrace{0^r10^N}_w\text{ , gdy }
				\left\lbrace \begin{matrix}
					p+q+r &=& N \\ p + q &\leqslant& N \\ 1 &\leqslant& q
				\end{matrix} \right.$$
				 Niech $i=0$ wówczas $z_0 = 0^p0^r10^N ~~ p+r < N \Rightarrow z_0 \not\in L$
			\end{itemize}
		\end{proof}
	\end{przyklad}	
	
	\begin{zad}
		Sprawdzić czy język słów binarnych $L = \{ w\in \{0,1\}^*: ~ \#_0w < \#_1w\}$\footnote{$\#$ oznacza liczbę wystąpień znaku z 
		indeksu dolnego w słowie}, jest regularny.
	\end{zad}
	\begin{zad}
		Sprawdzić czy język słów binarnych które składają się z sekwencji tych samych liter nie krótszych niż trzy, jest regularny.
	\end{zad}
	