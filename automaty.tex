\begin{df}[Automaty liniowo ograniczone]~\\
	Automaty liniowo ograniczone są to Maszyny Turinga z własnością stopu, które prowadzą obliczenia
	tylko na części taśmy będącej o długości równej wielokrotności długości słowa wejściowego.
	Dodatkowo maszyna ta musi być jednotaśmowa.
\end{df}

\begin{tw}
	Klasa języków akceptowanych przez automaty liniowo ograniczone zawiera się (bez równości) w
	klasie języków akceptowanych przez Maszyny Turinga z własnością stopu.
\end{tw}

\begin{tw}
	Automaty liniowo ograniczone akceptują języki kontekstowe.
	\begin{proof}
		Idea konstrukcji automatu równoważnego gramatyce nieograniczonej.
		Mając zadaną gramatykę nieograniczoną $G$:
		\begin{enumerate}
			\item Tworzymy 2-taśmową Maszynę Turinga
			\item Na pierwszej taśmie zapisujemy słowo wejściowe
			\item Na drugiej taśmie prowadzimy obliczenia według schematu
				\begin{enumerate}
					\item na drugiej taśmie umieszczamy symbol początkowy gramatyki
					\item w niedeterministyczny sposób wybieramy fragment słowa na drugiej taśmie
					\item do wybranego fragmentu stosujemy niedeterministycznie wybraną produkcję
				\end{enumerate}
			\item Sprawdzamy czy zawartość pierwszej taśmy jest równa zawartości drugiej taśmy i 
			jeśli tak to kończymy obliczenie akceptując słowo.
		\end{enumerate}
	\end{proof}
\end{tw}

\begin{tw}
	Gramatyki nieograniczone generują języki rekurencyjnie przeliczalne i są równoważne Maszyną Turinga
	\begin{proof}
		Idea konstrukcji automatu równoważnego gramatyce bezkontekstowej.
		Mając zadaną gramatykę kontekstową $G$:
		\begin{itemize}
			\item Tworzymy automat linowo ograniczony
			\item Reszta jest analogiczna do idei konstrukcji automatu równoważnego gramatyce
			nieograniczonej. Różnica polega na tym, że ze względu na ograniczenie długości
			taśmy oraz nieskracalność gramatyki obliczenie zawsze się skończy.
		\end{itemize}
	\end{proof}
\end{tw}

\begin{df}[Automaty ze stosem]
	Automatem ze stosem nazywamy system 
	\begin{equation}
		M = (Q, \Sigma, \Gamma, \delta, q_0, Z_0, F)
	\end{equation}
	gdzie $Z_0$ jest początkiem stosu a funkcja przejścia zdefiniowana jest następująco
	\begin{equation}
		\delta : Q \times \{\Sigma \cup \{ \varepsilon \} \}	\times \Gamma \rightarrow 
		\sum_{k=0}^\infty (Q\times\Gamma^\star)^k
	\end{equation}
\end{df}

\begin{tw}
	Klasa automatów deterministycznych ze stosem jest równoważna klasie automatów ze stosem niedeterministycznym
\end{tw}

\begin{tw}
	Automat ze stosem jest Maszyną Turinga, to znaczy można skonstruować Maszynę Turinga równoważną automatowi ze stosem
	\begin{proof}
		Pokażemy, że każdy automat ze stosem może być symulowany przez Maszynę Turinga z własnością stopu.
		\begin{itemize}
			\item Zakładamy, że dany jest automat ze stosem $M$
			\item Oznaczamy przez $M_T$ konstruowaną przez nasz Maszynę Turinga, która posiada dwie taśmy
			\item Pierwsza taśma symulować będzie taśmę wejściową automatu ze stosem
			\item Druga taśma symulować będzie stos
			\item Schemat ruchu (symulacji) o numerze $t$ automatu przez Maszynę Turinga
				\begin{enumerate}
					\item Jeśli druga taśma jest pusta, to jest to równoważne z pustym stosem. 
					Przechodzimy zatem do następnego stanu (być może akceptującego) 
					zgodnie z pewną funkcją przejścia
					\item Jeśli druga taśma nie jest pusta, odczytujemy po kolei wartości na drugiej taśmie
					pierwszą pozostawiając nienaruszoną i przechodzimy do kolejnych stanów uaktualniając nasz
					symulowany stos. Po uaktualnieniu stosu przechodzimy do następnego stanu wymazując
					wczytany znak na pierwszej taśmie. Musimy zadbać aby głowica drugiej taśmy wróciła na swoje
					miejsce.					
				\end{enumerate}
		\end{itemize}
	\end{proof}
\end{tw}

\begin{tw}
	Klasa automatów ze stosem jest równoważna klasie gramatyk bez kontekstowych
	\begin{proof}
	\begin{enumerate}
		\item Mając podaną gramatykę bez kontekstową $G$ transformujemy ją do postaci Geribach
		\item Konstruujemy automat z pustym stosem oraz jednym stanem. Który obliczamy następująco
			\begin{enumerate}
				\item Na wejściu umieszczamy analizowane słowo, a na stosie symbol początkowy gramatyki
				\item Na podstawie wczytanego znaku oraz nieterminala na wierzchu stosu niedeterministycznie
				wybieramy produkcję z nieterminala na szczycie stosu w ciąg symboli zaczynających się
				od wczytanego znaku.
				Wybór polega na niedeterministycznym wyborze takiego przejścia w automacie aby wierzchołek
				stosu zastąpić ciągiem nieterminali z prawej strony wybranej produkcji. \label{punktB}
				\item Następnie wczytywany jest kolejny znak i przechodzimy do punktu \ref{punktB}
				\item Akceptujemy jeśli osiągnęliśmy konfigurację z pustym stosem i pustym wejściem, we
				wszystkich pozostałych przypadkach odrzucamy.
			\end{enumerate}
	\end{enumerate}		
	\end{proof}
\end{tw}

\begin{tw}[O równoważności akceptowania]~\\
	Klasa automatów ze stosem akceptujących stanem końcowym jest równoważna klasie automatów ze stosem
	akceptujących pustym stosem (i wejściem)
	\begin{proof}
		\begin{enumerate}
			\item Możemy za symulować automat akceptujący stanem ($M_1$) automatem akceptującym pustym stosem ($M_2$)
			w ten sposób że w momencie gdy stos zostanie wymazany automat
			symulujący automatycznie zmieni swój stan na akceptujący.
			\item $M_2$ może za symulować $M_1$ w ten sposób, że w momencie przejścia do stany akceptującego
			\footnote{dokładniej to jeśli posiada konfigurację identyczna z konfiguracją akceptującą $M_1$}
			przechodzi do specjalnego stanu $\overline{q}$
			\item W stanie $\overline{q}$ stos jest czyszczony i automat $M_2$ zaakceptuje słowo.
		\end{enumerate}
	\end{proof}
\end{tw}

\begin{df}[Automat skończony deterministyczny]~\\
	Automatem skończonym deterministycznym nazywamy strukturę 
	\begin{equation}
		M = (Q, \Sigma, \delta, q_0, F)
	\end{equation}
	gdzie funkcja przejścia $\delta$ jest funkcją całkowitą, co oznacza że dla każdego 
	stanu istnieje dokładnie jedno przejście przez każdy symbol alfabetu wejściowego
\end{df}

\begin{df}[Domknięcie funkcji przejścia automatu deterministycznego]~\\
	Domknięciem funkcji przejścia automatu skończonego deterministycznego nazywamy
	funkcję
	\begin{equation}
		\overline{\delta} : Q\times\Sigma^\star \rightarrow Q
	\end{equation}
	taką, że:
	\begin{eqnarray}
		(\forall q\in Q) &\overline{q}(q, \varepsilon)& = q \\
		(\forall q\in Q)(\forall a\in\Sigma)(\forall w\in\Sigma^\star) 
		&\overline{\delta}(q, wa)& = \delta(\overline{\delta}(q, w), a)
	\end{eqnarray}
	$\overline{\delta}$ określa zachowanie automatu dla słowa nad alfabetem $\Sigma$
\end{df}

\begin{df}[Automat skończony niedeterministyczny]~\\
	Automatem skończonym niedeterministycznym nazywamy system 
	$M = (Q, \Sigma, \delta, q_0, F)$ gdzie funkcja przejścia jest postaci 
	\begin{equation}
		\delta : Q\times\Sigma \rightarrow 2^Q
	\end{equation}
	oraz jest również funkcją całkowitą. Wartościami funkcji przejścia są nie pojedyncze stany,
	a podzbiory zbioru stanów $Q$
\end{df}

\begin{df}[Domknięcie funkcji przejścia automatu skończonego niedeterministycznego]~\\
	Domknięcie funkcji przejścia automatu skończonego niedeterministycznego jest funkcja
	\begin{equation}
		\overline{\delta} : Q\times\Sigma^\star \rightarrow 2^Q
	\end{equation}
	taka że
	\begin{eqnarray}
		(\forall q\in Q) &\overline{\delta}(q, \varepsilon)& = \{q\} \\
		(\forall q\in Q)(\forall a\in\Sigma)(\forall w\in\Sigma^\star) &\overline{\delta}(q, wa)& = \bigcup_{r\in\overline{\delta}(q,w)}\delta(r, a)
	\end{eqnarray}
	$\overline{\delta}$ określa zachowanie automatu dla słowa nad alfabetem $\Sigma$. Zastosowanie go do konfiguracji
	początkowej określa zbiór możliwych stanów, do których może przejść automat po wczytaniu słowa wejściowego.
\end{df}

\begin{df}[Domknięcie stanu automatu]~\\
	$\varepsilon$-domknięciem stanu $q$ (ozn. $\varepsilon-CL(q)$) nazywamy zbiór stanów
	\begin{equation}
		\varepsilon-CL(q) = \{p : p \in \delta(q, \varepsilon)\}
	\end{equation}
\end{df}

\begin{tw}
	Klasa automatów skończonych deterministycznych jest równoważna klasie automatów skończonych niedeterministycznych.
	\begin{proof}
		Wykażemy że dla każdego automatu jednej klasy umiemy skonstruować równoważny automat drugiej klasy.
		\begin{enumerate}
			\item Każdy automat deterministyczny jest szczególnym przypadkiem automatu niedeterministycznego. Pojedyncze
			stany należałoby jedynie zastąpić jednoelementowymi zbiorami stanów.
			\item Aby wykazać że automat skończony niedeterministyczny da się symulować automatem deterministycznym oznaczamy
			pierwszy jako $M$ a automat symulujący jako $M'$
			\begin{itemize}
				\item Stanem odpowiadającym zbiorowi stanów $\{q_{i1}, q_{i2}, \dots, q_{in}\}$ będzie $[q_{i1}, q_{i2}, \dots, q_{in}]$
				czyli stan etykietowany nazwami stanów ze zbioru w automacie niedeterministycznym.
				\item Funkcję przejścia automatu deterministycznego definiujemy jako sumę wartości funkcji dla stanów którymi etykietowany
				jest dany stan
				\begin{eqnarray}
					\delta'([q_0], x) = [q_{i1}, q_{i2}, \dots, q_{ij}] &\Leftrightarrow& \delta(q_0, x) = \{q_{i1}, q_{i2}, \dots, q_{ij}\} \\
					\delta'([q_{ij}, \dots, g_{ij}], a) = [q_{i1}, q_{ik}, \dots, q_{ij}] &\Leftrightarrow& \bigcup_{l=1}^{j} \delta(q_{il}, x) = \{p_{i1}, p_{i2}, \dots, p_{ik}\}
				\end{eqnarray}
			\end{itemize}
		\end{enumerate}
	\end{proof}
\end{tw}

\begin{tw}
	Klasa automatów skończonych jest równoważna klasie gramatyk regularnych i jest równoważna klasie wyrażeń regularnych.
\end{tw}

