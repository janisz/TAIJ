\section{Gramatyki}
	
	\begin{df}
	\href{http://pl.wikipedia.org/wiki/Gramatyka_formalna}{Gramatyka} to sposób opisu języka, czyli podzbioru zbioru wszystkich
	 słów nad danym alfabetem.
	\end{df}
	\begin{df}[Relacja wywodu bezpośredniego]
	$C(V \cup T)* \times (V \cup T)*$
	\end{df}
	
	\begin{tw}
	Dwa słowa są w relacji wywodu bezpośredniego gdy jedno powstało z drugiego w wyniku pewnej produkcji
	\end{tw}
	
	\begin{df}
	Relacja wywodu jest przechodnim domknięciem relacji wywodu bezpośredniego.
	\end{df}
	
	\begin{df}
	Językiem generowanym przez gramatykę $G$ nazywamy zbiór słów nad alfabetem terminali, które pozostają w relacji wywodu z 
	symbolem początkowym.
	$$ L(G) = \{ w\in T^\star: S \Rightarrow w \} $$
	\end{df}
	
	\begin{df}
	\href{http://pl.wikipedia.org/wiki/Gramatyka_regularna}{Gramatyki regularne} są to gramatyki prawostronnie liniowe lub 
	lewostronnie liniowe
	\end{df}
	
	\begin{df}
	Gramatykę nazywamy prawostronnie liniową $\Leftrightarrow$ wszystkie jej produkcje są postaci $\begin{matrix}
	A&\rightarrow&wB \\
	A&\rightarrow&w
	\end{matrix}$
	gdzie $A, B \in V, ~ w\in T^\star$\footnote{początkowe wielkie litery alfabetu oznaczają symbole nieterminalne, 
	natomiast początkowe małe litery alfabetu to symbole terminalne. Greckie litery oznaczają ciągi symboli terminalnych i nieterminalnych
	Końcowe litery alfabetu oznaczają ciągi terminali}	 
	\end{df}
	
	\begin{przyklad}
		Konstrukcja gramatyki generującej język liczb binarnych, bez nieznaczących zer \\
		$G = (V, P, T, S) \quad V = \{S\} \cup \{A\} \quad T = \{0, 1\}$\footnote{Jeśli nie określono inaczej to 
		za symbol początkowy przyjmujemy $S$}
		$$
		\begin{matrix}
		S \to 0 & S \to 1 & S \to 1A \\
		A \to 0A & A \to 1A & A \to 1 & A \to 0\\		
		\end{matrix}
		\Rightarrow
		S \to 0|1|1|A, \quad A \to 0|1|0A|1A
		$$
	\end{przyklad}		
	
	\begin{tw}
		Gramatyki regularne generują tylko języki regularne
	\end{tw}
		
\subsection{\href{http://pl.wikipedia.org/wiki/Gramatyka_bezkontekstowa}{Gramatyki bezkontekstowe}}
	
	\begin{df}[Gramatyka bezkontekstowa]~\\
		$G$ jest gramatyką bezkonktekstową $\Leftrightarrow$ lewa strona każdej produkcji gramatyki $G$ jest złożona z jednego symbolu
		nieterminalnego to znaczy ma produkcje postaci:
			$$
				(A, \alpha), \text{gdzie} A\in V, \alpha \in \{T,V\}\star
			$$
	\end{df}
	
	\begin{df}[Język bezkontekstowy]~\\
		Językami bezkontekstowymi będziemy nazywać języki generowane przez gramatyki bezkontekstowe i tylko takie języki		
		
		\begin{uwaga}
			Językiem generowanym przez gramatykę bezkontekstową jest zbiór słół który da się wywieść z symbolu początkowego gramatyki stosująć
			jej produkcje
		\end{uwaga}
	\end{df}
	
	\begin{df}[Drzewo wywodu w gramatyce bezkontekstowej]~\\
		Jest to drzewo o następujących własnościach
		\begin{itemize}
			\item Wierzchołki etykietowane sybolami terminalnymi, nieterminalnymi lub $\varepsilon$
			\item Korzeń etykietowany symbolem początkowym gramatyki
			\item Dla produkcji $A\rightarrow x_1x_2 \dots x_n$, w drzewie wierzchołek $A$ ma następniki 
			kolejno $x_1, x_2, ..., x_n$ i odwrotnie
			\item Jeżeli wierzchołek etykietowany jest słowem pustym ($\varepsilon$) to jest jedynym
			następnikiem swojego przodka
		\end{itemize}
	\end{df}

	
	\begin{przyklad}
		Wykazać że język palindromów jest generowany przez gramatykę bezkontekstową.
	\begin{proof}
		$w$ - plaindrom nad $\{0,1\}$ \\
		\begin{enumerate}
			\item Jeśli $w$ jest słowem pustym ($\epsilon$) lub jedną literą to jest generowane
			\item Wykażemy że jesłi palindrom długości nie większych niż $n$ są generowane w tej gramatyce to
			generowane są również palindromy od długości $n+1$ i $n+2$\\
		\end{enumerate}\footnote{ten dowód można przeprowadzić rozpatrując parzystość $n$}
	\end{proof}
	\end{przyklad}		
	
	\begin{tw}
	Klasa języków regularnych zawiera się w klasie języków bezkontekstowych
	\end{tw}
	
	\subsubsection{Metody upraszczania gramatyk bezkontekstowych}
	
	\begin{df}[Symbole beżużyteczne]~\\
		\begin{description}
			\item[niegenerujące] symbole nieterminalne z których nie da się wywieść ciągu symboli terminalnych (być może pustego)
			\item[nieosiągalne] sumbole którch nie da się wywieść z symbolu początkowego
		\end{description}				
	\end{df}	
	
	\begin{lemat}[Usuwanie symboli bezużytecznych 1]\footnote{Uzasadnienie lematu można przeprowadzić alanlizuąć drzewa wywodu grmatyk}~\\
		Dla dowolnej gramatyki bezkontekstowej $G = (V, T, P, S)$ genrującej niepusty język $L(G)$ istnieje równoważna gramatyka
		bezkontekstowa $G' = (V', T', P', S')$ tak że z każdego symbolu nieterminalnego z $V'$ można wyprowadzić ciąg symboli terminalnych (być może pusty)
	\end{lemat}	
	
	\begin{lemat}[Usuwanie symboli bezużytecznych 2]\footnote{Uzasadnienie lematu można przeprowadzić alanlizuąć drzewa wywodu grmatyk}~\\
		Dla dowolnej gramatyki bezkontekstowej $G = (V, T, P, S)$ genrującej niepusty język $L(G)$ istnieje równoważna gramatyka
		bezkontekstowa $G' = (V', T', P', S')$ tak że każdy symbol terminalny i nieterminalny z $T' \cup V'$ można wyprowadzić
		z  symbolu startowego $S$
	\end{lemat}		
	
	\begin{alg}[Usuwanie symboli niegenerujących]~\\
		\begin{enumerate}
			\item Wszystkie terminale oznaczamy jako generujące
			\item Symbole (nieterminale) mające po prawej stronie produkcji symbol/e generujące oznaczamy jako generujące
			\item Powtarzamy punkt 2 dopóki zbiór symboli generujących nie będzie stabilny
			\item Usuwamy wszystkie symbole nie oznaczone jako generujące oraz produkcje które je generują
		\end{enumerate}
	\end{alg}
	
	\begin{alg}[Usuwanie symboli nieosiągalnych]~\\
		\begin{enumerate}
			\item Oznaczamy symbol początkowy gramatyki i sumbole przez niego generowane (po prawej stronie produkcji)
			jako osiągalne
			\item Symbole generowane z symboli osiągalnych oznaczamy jako osiągalne
			\item Powtarzamy punk 2 dopóki zbiór symboli osiągalnych nie będzie stabilny
			\item Usuwamy wszystkie symvole nie oznaczone jako osiągalne oraz produkcje które je zawierają
		\end{enumerate}
	\end{alg}	
	
	\begin{tw}
		Po zastosowani lematów w podanej kolejności uzyskamy gramatykę bez symboli bezużytecznych
	\end{tw}
	
	\begin{df}
		Symbol nieterminalny nazywamy wycieralnym $\Leftrightarrow$ można z niego wywieść ciąg pusty
	\end{df}		

	\begin{df}
		Produkcję nazywamy wycieralną jeśli prawą strona tej produkcji można przekształcić do symbolu pustego
	\end{df}			
	
	\begin{alg}[Metoda wyznaczania symboli wycieralnych]\footnote{Uzasadnienie poprawności tej metody polega na 
	obserwacji własności wywodu słow w obu grmatykach -- przed i po usuwaniu}~\\
		\begin{enumerate}
			\item Przeglądamy produkcję i za symbole wycieralne uznajemy te symbole które po lewej stronie produkcji 
			prowadzą w słowo puste
			\item Dopóki zbiór symboli wycieralnych się zmienia
			\begin{enumerate}
				\item przeglądamy wszystkie produkcje i za symbole wycieralne uznajemy te które po są po lewej stronie
				produkcji a po prawej ciągi symboli już uznanych za wycieralne
			\end{enumerate}
		\end{enumerate}
	\end{alg}
	
	\begin{df}[Usuwanie symboli bezużytecznych i wycieralnych]~\\
		Jeśli $G = (V, T, P, S)$ jest gramatyką bezkontekstową i bez symboli bezużytecznych i nie generuje słowa pustego
		to można znaleźć równoważną jej gramatykę bez symboli bezużytecznych i wycieralnych.\\
		\emph{Metoda przekształcania}\\
		Załóżmy że $A \to X_1,X_2,\dots,X_n$ jeśli któryś z symboli $X_i$ jest wycieralny to tę produkcję zastępujemy produkcjami
		$\begin{matrix}
			A \to& X_1,X_2,\dots,X_{i-1},X_i,X_{i+1},X_n \\ A \to& X_1,X_2,\dots,X_{i-1},X_{i+1},X_n
		\end{matrix}$
		z gramatyki zawsze usuwamy produkcje w słowo puste
	\end{df}	
	
	\begin{lemat}[Produkcje jednostkowe]~\\
		Niech $G$ będzie gramatyką bezkontekstową bez symboli bezużytecznych oraz wycieralnych. Wówczas 
		istnieje rółnoważna jej gramatyka $G'$ również bez produkcji jednostkowych
	\end{lemat}
	
	\begin{alg}[Usuwanie produkcji jednostkowych]\footnote{Uzasadnienie jest analogiczne do algorytmu usuwania symboli wycieralnych}~\\
		\begin{enumerate}
			\item Usuwamy wszystkie produkcje w ten sam nieterminal
			\item Produkcje z nieterminala $A$ w inny nieterminal $B$ zamieniamy na produkcję z $A$ w prawą stronę produkcji z $B$
		\end{enumerate}
	\end{alg}
	
\subsection{Postaci grmatyk bezkontekstowych}

	\begin{tw}
		Jeśli $L$ jest jezykiem bezkontekstowym, to jezyk $L' = L - \varepsilon$ też jest językiem bezkontekstowym
	\end{tw}
	
	\begin{df}[\href{http://pl.wikipedia.org/wiki/Posta\%C4\%87_normalna_Chomsky\%27ego}{Postać normalna Chomskiego}]~\\
		Gramatyka $G$ jest w postaci normalnej Chomskiego $\Leftrightarrow$ gdy wszyskieg jej produkcje są postaci:
		\begin{itemize}
			\item $A\rightarrow BC$
			\item $A \rightarrow \alpha$
		\end{itemize}
		gdzie $A,B,C \in V$ (symbole nieterminalne), $\alpha \in T$ (symbol terminalny)
	\end{df}
	
	\begin{lemat}
		Każdą gramatykę bezkontekstową, która nie generuje słowa pustego można przekształcić do postaci Chomskiego
	\end{lemat}
	
	\begin{alg}[Przekształcanie do postaci Chomskiego]~\\
		\begin{enumerate}
			\item Usuwamy z gramatyki symbole bezużyteczne, wycieralne i produkcje jednostkowe
			\item Jeżeli po prawej stronie występuje więcej niż jeden (1) symbol, to wszystkie
			terminale tam występujące zamieniamy na nowe nieterminale i dodajemy
			nowe produkcje z nowych nieterminali w odpowiadające im terminale
			\item Wszytkie pozostałe produkcje w ciąg nieterminali długości większej niż dwa (2)
			modyfikujemy w sposób następujący
				\begin{enumerate}
					\item Dopóki ciąg nieterminali po prawej stronie produkcji nie zmniejszy się do długości dwa (2)
					zastępujemy jakąś parę nieterminali z tego ciągu nowym nieterminalem i dodajemy nową produkcję
					z nowego nieterminala w dwa stare, usunięte nieterminale
				\end{enumerate}
		\end{enumerate}
	\end{alg}
	
	\begin{uwaga}
		Drzewo wywodu słowa w gramatyce bezkontekstowej w postaci Chomskiego jest drzewem binarnym
	\end{uwaga}
	
	\begin{df}[\href{http://pl.wikipedia.org/wiki/Posta\%C4\%87_normalna_Greibach}{Postać noramlna Greibach}]~\\
		Gramatka bezkontekstowa $G$ jest w postaci Greibach $\Leftrightarrow$ wszystkie jej produkcje są postaci:
		$$
			A \rightarrow a\alpha
		$$
		gdzie $a\in T$, $\alpha \in V\star$ (ciąg nieterminali)
	\end{df}
	
	\begin{lemat}
		Każdą gramatykę bezkontekstową nie generującą słowa pustego da się sporowadzić do postaci Greibach
	\end{lemat}
	
	\begin{alg}[Przekształcanie do postaci Greibach]~\\
		\begin{enumerate}
			\item Sprowadzamy gramatykę do postaci Chomskiego
			\item Numerujemy symbole nieterminalne (ustawiając je w ciąg $V = \{A_1, A_2, \dots, A_n\}$
			\item Otrzymane $A_i$ produkcje spełniają warunek, że dla $i=1,\dots, k-1$ produkcje z $A_i$ są postaci
				\begin{itemize}\label{warunek}
					\item $A_i \rightarrow a\alpha$, gdzie $a\in T$ i $\alpha\in V\star$
					\item $A_i \rightarrow A_j\alpha$, gdzie $j > i$ 
					oraz $A_j$ jest nieterminalem z naszego ciągu o indeksie $j$ i $\alpha \in V\star$
				\end{itemize}
			Ponadto $A_k$ nie spełnia tego warunku wiec mamy dwa przypadki
				\begin{enumerate}
					\item $A_k \rightarrow A_j\alpha$ oraz $j < k$\label{1}
					\item $A_k \rightarrow A_k\alpha$\label{2}
				\end{enumerate}
			\item W przypadku \ref{1} zastępujemy nieterminale $A_j$ prawymi stronami $A_j$-produkcji
			i postępujemy tak aż do wyeliminowania problemu, po czym wracamy do punktu 3, chyba że $k = n$
			\item W przypadku gdy natrafiliśmy na sytuację \ref{2} usuwamy problem w sposób następujący:
				\begin{enumerate}
					\item Produkcję $A_k \rightarrow A_k\alpha_1|A_k\alpha_2|\dots|A_k\alpha_m|\beta_1|\dots|\beta_n$
					zastępujemy produkcjami:
					$$
						A_k \rightarrow \beta_1|\dots|\beta_n|\beta_1B_k|\dots|\beta_nB_k \\
						B_k \rightarrow \alpha_1|\dots|\alpha_m|\alpha_1B_k|\dots|\alpha_mB_k
					$$, gdzie $B_k$ jest nowym nieterminalem
					\item Stosujac punkt 4 i 5 zapewniliśmy spełnienie warunku \ref{warunek} przez $A_k$ i możemy zająć się produkcją $A_{k+1}$
					\item Powtarzamy czynność z punktu 3 i w miarę potrzeb 4 i 5 aż wszystkie $A_i$ nie będą spełniały \ref{warunek}
				\end{enumerate}
			\item Wszystkie produkcje $A_i \rightarrow a\alpha$ są już w postaci Greibach natomiast produkcje pozostałe które są w postaci
			$A_i\rightarrow A_j\alpha$, $j > i$ modyfikujemy zamieniając $A_j$ na prawą stronę produkcji $A_j$
		\end{enumerate}
	\end{alg}
	
\subsubsection{Sprawdzanie czy język jest bezkontekstowy}

	\begin{df}[Niejednoznaczność gramatyki]~\\
		Gramatykę nazywamy niejednoznaczną $\Leftrightarrow$ istnieje takie słowo dla którego istnieje więcej niż jedno drzewo wywodu		
	\end{df}
	
	\begin{lemat}[\href{http://pl.wikipedia.org/wiki/Lemat_o_pompowaniu_dla_j\%C4\%99zyk\%C3\%B3w_bezkontekstowych}{O pompowaniu (dla języków bezkontekstowych)}]~\\
		Jeśli $L$ jest językiem bezkontekstowym, to istnieje stała $n_L$ taka że dla
		dowolnego słowa $z\in L$ spełniony jest warunek:
		$$
			(|z| \geqslant n_L) \Rightarrow [( \exists_{u,v,w,x,y} z = uvwxy \wedge |vwx| \leqslant n_L \wedge |vx| \geqslant 1) \forall_{i=0,1,\dots} z_i = uv^iwx^iy \in L]
		$$
		\begin{uwaga}
			Lemat o pompowaniu dla języków regularnych jest szczególnym przypadkiem tego lematy
		\end{uwaga}
		
		\begin{proof}~\\
			\begin{enumerate}
				\item Jeżeli $L$ jest skończony, to lemat jest prawdzimy (wystarczy przyjąć $n_L$ wieksze niż długość najdłuższego słowa)
				\item Język jest nieskończony
					\begin{enumerate}
						\item Skoro język jest bezkontekstowy to weźmy gramatykę $G$ w postaci Chomskiego generującą ten język. Wtedy 
						drzewo wywodu słów z $L$ będą drzewami binarnymi. Drzewo wywodu mające $k$ liści ma zatem $log_2k$ poziomów
						(wysokość $h = log_2k$). Zatem liczba krawędzi między ścieżkami z korzenia do liścia jest nie mniejsza niż $h$.
						Wierzchołki ścieżki oprócz ostatniego etykietujemy symbolami nieterminalnymi a ostatni sybolem terminalnym.
						\item Niech $|V| = n$ (mamy $n$ symboli nieterminalnych), wtedy przynajmniej $n_L = 2^{n+1}$, a co za tym 
						idzie drzewo wywodu słowa $z$ o długości nie mniejszej niż $n_L$ będzie miało wysokość przynajmniej $n+1$
						Wynika z tego że na ścieżce korzeń-liść pojawią się przynajmniej dwa wierzchołki etykietowane tym samym terminalem
						\item Oznaczamyu przez $A_i$ najbliższe wystąpienie powtarzającego się nieterminala licąć od liścia, a przez $A_k$
						jego drugie wystąpienie. Oznaczamy przez $w$ plon poddrzewa $A_l$ oraz przez $v$ i $x$ części plony poddrzewa $A_k$
						pozostałe po wyłączeniu w. Oznaczamy przez $u$ i $y$ części plony całego drzewa pozostałę po wyłączeniu $vwx$. 
						Poprez takie oznaczenia podzieliliśmy słowo $z$ na pięć części $z = uvwx$. Z wyboru najbliższej pary
						powtarzających się nieterminali wynika że $|vwx| \leqslant n_L$, bo długość ścieżki jest nie większa $n+1$
						($A$ wybieramy dwukrotnie, a pozostałe nieterminale najwyżej jednokrotnie)
						\item Dla $i=0$ z lematu równoważnym jest zastąpienie poddrzewa $A_k$ poddrzewem $A_L$ takie drzewo wywodu jest 
						nadal poprawnym drzewem wywody w gramatyce $G$, a na dodatek generuje słowo $z_0 = uwy$
						\item Ostatnią rzeczą którą trzeba pokazać to że dla $i = k \geqslant 2$  z lematu równoważnym jest $k$-krotnemu
						zastąpieniu poddrzewa $A_L$ przez poddrzewo $A_\alpha$. Znowy otrzymamy drzewo wywodu w gramatyce $G$, tym
						razem generujące słowo $z_k = uv^kwx^ky$.
					\end{enumerate}					 
			\end{enumerate}
		\end{proof}				
	\end{lemat}
			
	\begin{lemat}[Zaprzeczenie lematu o pompowaniu]~\\
		Jeśli dla dowolnej stałej $n\in\mathbb{N}$ istnieje $z\in L$ takie że
		$$
			(|z| \geqslant n_L) \Rightarrow [( \forall_{u,v,w,x,y} z = uvwxy \wedge |vwx| \leqslant n_L \wedge |vx| \geqslant 1) \exists_{i=0,1,\dots} z_i = uv^iwx^iy \not\in L]
		$$
		to język $L$ nie jest językiem bezkontekstowym
	\end{lemat}			
		
	\begin{lemat}[\href{http://pl.wikipedia.org/wiki/Lemat_Ogdena}{Ogdena}]~\\
		Jeśli $L$ jest językiem bezkontekstowym, to istnieje stała $n_L$ taka, że dla dowolnego słowa z języka $L$, w którym wyróżniono przynajmniej
		$n_L$ symboli spełniony jest warunek:\\
		Istnieje taki podział słowa $z$ na pięć części $z = uvwxy$ takich że:
		\begin{enumerate}
			\item $vx$ zawiera przynajmniej jeden wyróżniony symbol
			\item $vwx$ zawiera co najwyżej $n_L$ wyróżnionych symboli
			\item $(\forall i\geqslant 0) \quad uv^iwx^iy \in L$
		\end{enumerate}
		
		\begin{proof}
			Dowód jest analogiczny do dowodu lematu o pompowaniu, z tym że konstrukcja ścieżki ulega drobnej modyfikacji.
			Podczas konstrukcji ścieżki wybieramy następnika którego poddrzewo o korzeniu w tym wierzchołku zawiera nie
			mniej wyróżnionych symboli, niż poddrzewo drugiego następnika
		\end{proof}			
		
		\begin{uwaga}
			Lemat o pompowaniu jest szczególnym przypadkiem lematu Ogdena
		\end{uwaga}
	\end{lemat}	
	
	\begin{lemat}[Zaprzeczenie lematu Ogdena]~\\
		Jeśli dla $\forall n\in \mathbb{N}$ istnieje słowo $z \in L$ w którym wyróżniono co najmniej $n$ symboli oraz
		dla każdego podziału słowa na pięć części takiego że
		\begin{itemize}
			\item $vx$ zawiera przynajmniej jeden wyróżniony symbol
			\item $vwx$ zawiera co najwyżej $n$ wyróżnionych symboli
		\end{itemize}
		$$
		(\exists i=1,2,\dots) \quad uv^iwx^iy \in L
		$$
		to język $L$ nie jest językiem bezkontekstowym
	\end{lemat}			
		
