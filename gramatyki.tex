	\section{Gramatyki}
	
	\begin{df}
	\href{http://pl.wikipedia.org/wiki/Gramatyka_formalna}{Gramatyka} to sposób opisu języka, czyli podzbioru zbioru wszystkich
	 słów nad danym alfabetem.
	\end{df}
	\begin{df}[Relacja wywodu bezpośredniego]
	$C(V \cup T)* \times (V \cup T)*$
	\end{df}
	
	\begin{tw}
	Dwa słowa są w relacji wywodu bezpośredniego gdy jedno powstało z drugiego w wyniku pewnej produkcji
	\end{tw}
	
	\begin{df}
	Relacja wywodu jest przechodnim domknięciem relacji wywodu bezpośredniego.
	\end{df}
	
	\begin{df}
	Językiem generowanym przez gramatykę $G$ nazywamy zbiór słów nad alfabetem terminali, które pozostają w relacji wywodu z 
	symbolem początkowym.
	$$ L(G) = \{ w\in T^\star: S \Rightarrow w \} $$
	\end{df}
	
	\begin{df}
	\href{http://pl.wikipedia.org/wiki/Gramatyka_regularna}{Gramatyki regularne} są to gramatyki prawostronnie liniowe lub 
	lewostronnie liniowe
	\end{df}
	
	\begin{df}
	Gramatykę nazywamy prawostronnie liniową $\Leftrightarrow$ wszystkie jej produkcje są postaci $\begin{matrix}
	A&\rightarrow&wB \\
	A&\rightarrow&w
	\end{matrix}$
	gdzie $A, B \in V, ~ w\in T^\star$\footnote{początkowe wielkie litery alfabetu oznaczają symbole nieterminalne, 
	natomiast początkowe małe litery alfabetu to symbole terminalne. Greckie litery oznaczają ciągi symboli terminalnych i nieterminalnych
	Końcowe litery alfabetu oznaczają ciągi terminali}	 
	\end{df}
	
	\begin{przyklad}
		Konstrukcja gramatyki generującej język liczb binarnych, bez nieznaczących zer \\
		$G = (V, P, T, S) \quad V = \{S\} \cup \{A\} \quad T = \{0, 1\}$\footnote{Jeśli nie określono inaczej to 
		za symbol początkowy przyjmujemy $S$}
		$$
		\begin{matrix}
		S \to 0 & S \to 1 & S \to 1A \\
		A \to 0A & A \to 1A & A \to 1 & A \to 0\\		
		\end{matrix}
		\Rightarrow
		S \to 0|1|1|A, \quad A \to 0|1|0A|1A
		$$
	\end{przyklad}		
	
	\begin{tw}
		Gramatyki regularne generują tylko języki regularne
		
\subsection{\href{http://pl.wikipedia.org/wiki/Gramatyka_bezkontekstowa}{Gramatyki bezkontekstowe}}
	\end{tw}		
		\begin{tw}
		Gramatyki bezkontekstowe generują tylko języki bezkontekstowe
	\end{tw}		
	
	\begin{przyklad}
		Wykazać że język palindromów jest generowany przez gramatykę bezkontekstową.
	\begin{proof}
		$w$ - plaindrom nad $\{0,1\}$ \\
		\begin{enumerate}
			\item Jeśli $w$ jest słowem pustym ($\epsilon$) lub jedną literą to jest generowane
			\item Wykażemy że jesłi palindrom długości nie większych niż $n$ są generowane w tej gramatyce to
			generowane są również palindromy od długości $n+1$ i $n+2$\\
		\end{enumerate}\footnote{ten dowód można przeprowadzić rozpatrując parzystość $n$}
	\end{proof}
	\end{przyklad}		
	
	\begin{tw}
	Klasa języków regularnych zawiera się w klasie języków bezkontekstowych
	\end{tw}
	
	\subsubsection{Metody upraszczania gramatyk}
	
	\begin{lemat}[Usuwanie symboli bezużytecznych 1]
		Dla dowolnej gramatyki bezkontekstowej $G = (V, T, P, S)$ genrującej niepusty język $L(G)$ istnieje równoważna gramatyka
		bezkontekstowa $G' = (V', T', P', S')$ tak że z każdego symbolu nieterminalnego z $V'$ można wyprowadzić ciąg symboli terminalnych (być może pusty)
	\end{lemat}	
	
	\begin{lemat}[Usuwanie symboli bezużytecznych 2]
		Dla dowolnej gramatyki bezkontekstowej $G = (V, T, P, S)$ genrującej niepusty język $L(G)$ istnieje równoważna gramatyka
		bezkontekstowa $G' = (V', T', P', S')$ tak że każdy symbol terminalny i nieterminalny z $T' \cup V'$ można wyprowadzić
		z  symbolu startowego $S$
	\end{lemat}		
	
	\begin{tw}
		Po zastosowani lematów w podanej kolejności uzyskamy gramatykę bez symboli bezużytecznych
	\end{tw}
	
	\begin{df}
		Symbol nieterminalny nazywamy wycieralnym $\Leftrightarrow$ można z niego wywieść ciąg pusty
	\end{df}		

	\begin{df}
		Produkcję nazywamy wycieralną jeśli prawą strona tej produkcji można przekształcić do symbolu pustego
	\end{df}			
	
	\begin{df}Metoda wyznaczania symboli wycieralnych\\
		\begin{enumerate}
			\item Przeglądamy produkcję i za symbole wycieralne uznajemy te symbole które po lewej stronie produkcji 
			prowadzą w słowo puste
			\item Dopóki zbiór symboli wycieralnych się zmienia
			\begin{enumerate}
				\item przeglądamy wszystkie produkcje i za symbole wycieralne uznajemy te które po są po lewej stronie
				produkcji a po prawej ciągi symboli już uznanych za wycieralne
			\end{enumerate}
		\end{enumerate}
	\end{df}
	
	\begin{df}
		Jeśli $G = (V, T, P, S)$ jest gramatyką bezkontekstową i bez symboli bezużytecznych i nie generuje słowa pustego
		to można znaleźć równoważną jej gramatykę bez symboli bezużytecznych i wycieralnych.\\
		\emph{Metoda przekształcania}\\
		Załóżmy że $A \to X_1,X_2,\dots,X_n$ jeśli któryś z symboli $X_i$ jest wycieralny to tę produkcję zastępujemy produkcjami
		$\begin{matrix}
			A \to& X_1,X_2,\dots,X_{i-1},X_i,X_{i+1},X_n \\ A \to& X_1,X_2,\dots,X_{i-1},X_{i+1},X_n
		\end{matrix}$
		z gramatyki zawsze usuwamy produkcje w słowo puste
	\end{df}	